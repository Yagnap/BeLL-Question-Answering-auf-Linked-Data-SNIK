\chapter{Klassifizierung der Textbuchfragen}\label{ch:klassifizierungtextbuchfragen}

\section{Erste Einordnung}\label{sub:firsteval}

Die erste Einordnung der Fragen aus \citet{bb} erfolgt hier\footnote{Basierend auf \citet{arneba}.
Als CSV verfügbar unter:\\\url{https://github.com/Yagnap/BeLL-Question-Answering-auf-Linked-Data-SNIK/blob/main/Data/Tabellen/Anhang/fragenklassifikation.csv}}:

\begin{longtable}{c p{6.5 cm} c c c c}
    \caption[Fragenklassifikation]{Klassifizierung der Fragen
    aus \citet{bb} basierend auf \citet{arneba}.
    Fett: Unterschiedliche Klassifizierung bezüglich der Eignung.
    Komma bei Frageart: Teilfragen haben unterschiedliche Frageart.
    Fragetyp und -art nach Anfangsbuchstaben abgekürzt.}
    \label{tab:fragenklassifikation}
    \\
    \toprule
    \rot{\textnormal{Kapitel/ID}}&\rot{\textnormal{Frage}}&\rot{\textnormal{Fragetyp}}&\rot{\textnormal{Frageart}}&\rot{\textnormal{Eignung}}&\rot{\textnormal{Orginal}} \\
    \midrule
    \endfirsthead
    \toprule
    \rot{\textnormal{Kapitel/ID}}&\rot{\textnormal{Frage}}&\rot{\textnormal{Fragetyp}}&\rot{\textnormal{Frageart}}&\rot{\textnormal{Eignung}}&\rot{\textnormal{Orginal}} \\
    \midrule
    \endhead
    1/1 & Why is systematic information processing in health care institutions important? & K & S & \xmark & \xmark \\
    1/2 & What are appropriate models for health information systems? & F & S & \cmark & \cmark \\
    1/3 & How do health information systems look like and what architectures are appropriate? & P, F & Z & \xmark & \xmark \\
    1/4 & How can we assess the quality of health information systems? & P & S & \xmark & \xmark \\
    1/5 & How can we strategically manage health information systems? & P & S & \xmark & \xmark \\
    1/6 & How can good information systems be designed and maintained? & P & S & \xmark & \xmark \\
    2/1 & What is the significance of information systems for health care? & K & S & \xmark & \xmark \\
    2/2 & How does technical progress affect health care? & P & S & \xmark & \xmark \\
    2/3 & Why is systematic information management important? & K & S & \xmark & \xmark \\
    3/1 & What is the difference between data, information and knowledge? & V & Z & \xmark & \xmark \\
    3/2 & What are information systems, and what are their components? & F & Z & \xmark & \xmark \\
    3/3 & What is information management? & F & S & \cmark & \cmark \\
    4/1 & What are hospital information systems? & F & S & \cmark & \cmark \\
    4/2 & What are transinstitutional health information systems? & F & S & \cmark & \cmark \\
    4/3 & What are challenges for health information systems? & F & S & \cmark & \cmark \\
    4/4 & What are electronic health records? & F & S & \cmark & \cmark \\
    5/1 & What are models, metamodels and reference models? & F & Z & \xmark & \xmark \\
    5/2 & What are typical metamodels for modeling various aspects of HIS? & F & S & \cmark & \cmark \\
    5/3 & What is 3LGM$^2$? & F & S & \cmark & \cmark \\
    5/4 & What are typical reference models for HIS? & F & S & \cmark & \cmark \\
    6/1 & What kind of data has to be processed in hospitals? & F & S & \cmark & \cmark \\
    6/2 & What are the main hospital functions? & F & S & \cmark & \cmark \\
    6/3 & What are the typical information processing tools in hospitals? & F & S & \cmark & \cmark \\
    6/4 & What are different architectures of HIS? & F & S & \cmark & \cmark \\
    6/5 & How can integrity and integration be achieved within HIS? & P & S & \xmark & \xmark \\
    6.4/1 & What application components are used in hospitals, and what are their characteristics? & F & Z & \xmark & \xmark \\
    \textbf{6.5/1} & \textbf{How can architectures of HIS be categorized?} & \textbf{F} & \textbf{S} & \cmark & \xmark \\
    6.5/2 & What differs integrity from integration? & V & Z & \xmark & \xmark \\
    \textbf{6.5/3} & \textbf{What standards and technologies are available to support integration of HIS?} & \textbf{F} & \textbf{Z} & \xmark & \cmark \\
    6.5/4 & How can integration efforts be reduced by decreasing the variety of application components in a HIS? & P & K & \xmark & \xmark \\
    6.6/1 & What computer-based and non-computer-based physical data processing systems can be found in hospitals? & F & Z & \xmark & \xmark \\
    6.6/2 & What is meant by the term \enquote{infrastructure}? & F & S & \cmark & \cmark \\
    6.7/1 & How can physical data processing systems be grouped and arranged in order to support application components in an optimal way? What architectures can this result? & P, F & Z & \xmark & \xmark \\
    6.7/2 & What is meant by physical integration? & F & S & \cmark & \cmark \\
    6.7/3 & How do modern computing centers look like? & P & S & \xmark & \xmark \\
    7/1 & How do architectures of transinstitutional health information systems differ from those of hospital information systems? & V & Z & \xmark & \xmark \\
    7/2 & What additional challenges do we have to cope with? & F & S & \xmark & \xmark \\
    \textbf{7/3} & \textbf{Which strategies are appropriate for maintaining electronic health records in a transinstitutional health information system?} & \textbf{F} & \textbf{K} & \cmark & \xmark \\
    8/1 & Which facets of quality have to be considered in HIS? & F & S & \cmark & \cmark \\
    \textbf{8/2} & \textbf{What are the characteristics of the quality of structures in HIS?} & \textbf{F} & \textbf{S} & \cmark & \xmark \\
    \textbf{8/3} & \textbf{What are the characteristics of the quality of processes of HIS?} & \textbf{F} & \textbf{S} & \cmark & \xmark \\
    \textbf{8/4} & \textbf{What are the characteristics of quality of outcome of HIS?} & \textbf{F} & \textbf{S} & \cmark & \xmark \\
    8/5 & What does information management have to balance in order to increase the quality of a HIS? & K & S & \xmark & \xmark \\
    \textbf{8/6} & \textbf{How can quality of HIS be evaluated?} & \textbf{F} & \textbf{K} & \cmark & \xmark \\
    \textbf{8.2/1} & \textbf{What criteria for quality of data exist?} & \textbf{F} & \textbf{S} & \cmark & \xmark \\
    8.2/2 & What criteria for computer-based application components and physical data processing systems exist? & F & Z & \xmark & \xmark \\
    \textbf{8.2/3} & \textbf{What criteria for the overall HIS architecture exist?} & \textbf{F} & \textbf{S} & \cmark & \xmark \\
    \textbf{8.3/1} & \textbf{What are the characteristics of the quality of processes of HIS?} & \textbf{F} & \textbf{S} & \cmark & \xmark \\
    \textbf{8.4/1} & \textbf{What are the characteristics of quality of outcome of HIS especially in hospitals?} & \textbf{F} & \textbf{S} & \cmark & \xmark \\
    8.5/1 & What does information management have to balance in order to increase the quality of health information systems? & K & S & \xmark & \xmark \\
    \textbf{8.6/1} & \textbf{What are major phases of an IT evaluation study?} & \textbf{F} & \textbf{S} & \cmark & \xmark \\
    \textbf{8.6/2} & \textbf{What are major IT evaluation methods?} & \textbf{F} & \textbf{S} & \cmark & \xmark \\
    9/1 & What does information management mean and how can strategic, tactical and operational information management be differentiated? & F, V & Z & \xmark & \xmark \\
    \textbf{9/2} & \textbf{What organizational structures are appropriate for information management in hospitals?} & \textbf{F} & \textbf{S} & \cmark & \xmark \\
    \textbf{9/3} & \textbf{What are the tasks and methods for strategic HIS planning?} & \textbf{F} & \textbf{K} & \cmark & \xmark \\
    \textbf{9/4} & \textbf{What are the tasks and methods for strategic HIS monitoring?} & \textbf{F} & \textbf{K} & \cmark & \xmark \\
    \textbf{9/5} & \textbf{What are the tasks and methods for strategic HIS directing?} & \textbf{F} & \textbf{K} & \cmark & \xmark \\
    9/6 & How can experts for information management in hospitals be gained? & P & S & \xmark & \xmark \\
    9.2/1 & What does information management in general and in hospitals encompass? & K & K & \xmark & \xmark \\
    9.2/2 & What are the three main scopes of information management? & F & K & \cmark & \cmark \\
    9.2/3 & What are the tasks of strategic, tactical and operational information management in hospitals? & F & Z & \xmark & \xmark \\
    9.2/4 & What is meant by IT service management and how is it related to information management? & F & Z & \xmark & \xmark \\
    9.3.4.2/1 & Which organizational units are involved in information management? & F & K & \cmark & \cmark \\
    9.3.4.2/2 & Which boards and persons are involved in information management? & F & section & \cmark & \cmark \\
    9.3.4.2/3 & Who is responsible for strategic information management? & F & S & \cmark & \cmark \\
    9.3.4.2/4 & Who is responsible for tactical information management? & F & S & \cmark & \cmark \\
    9.3.4.2/5 & Who is responsible for operational information management? & F & S & \cmark & \cmark \\
    9.3.4.2/6 & Who is the CIO, and what is his or her responsibility? & F & Z & \xmark & \xmark \\
    9.4/1 & What are the typical tasks for strategic HIS planning? & F & K & \cmark & \cmark \\
    9.4/2 & What are the typical methods for strategic HIS planning? & F & K & \cmark & \cmark \\
    9.4/3 & What is the goal and typical structure of a strategic information management plan? & F & Z & \xmark & \xmark \\
    9.5/1 & What are the typical tasks of strategic HIS monitoring? & F & S & \cmark & \cmark \\
    9.5/2 & What are the typical methods of strategic HIS monitoring? & F & S & \cmark & \cmark \\
    9.6/1 & What are the typical tasks of strategic HIS directing? & F & S & \cmark & \cmark \\
    9.6/2 & What are the typical methods of strategic HIS directing? & K & S & \cmark & \cmark \\
    10/1 & What are health care networks? & F & S & \cmark & \cmark \\
    \textbf{10/2} & \textbf{How can health care networks be described?} & \textbf{F} & \textbf{S} & \cmark & \xmark \\
    \textbf{10/3} & \textbf{What organizational structures are appropriate for information management in health care networks?} & \textbf{F} & \textbf{S} & \cmark & \xmark \\
    10/4 & How can good information systems be maintained? & P & S & \xmark & \xmark \\
  
    \bottomrule \\
  \end{longtable}
  
\section[Formulierung der SPARQL-Abfragen]{Formulierung der \ac{sparql}-Abfragen}\label{sub:sparqltextbuchfragen}

Im Folgenden wurden die natürlichsprachigen Fragen mit \ac{sparql}-Abfragen beantwortet\footnote{Als CSV verfügbar unter:\\\url{https://github.com/Yagnap/BeLL-Question-Answering-auf-Linked-Data-SNIK/blob/main/Data/Tabellen/Anhang/formulierung\_sparql\_bb\_textbuchfragen.csv}}:

\begin{lstlisting}[language=SPARQL]
    # 1/2 What are appropriate models for health information systems?
    SELECT DISTINCT ?s1
    WHERE
      { ?s1 rdfs:subClassOf+ bb:Model . }
    
    # 3/3 What is information management?
    SELECT DISTINCT ?o1
    WHERE
      {  VALUES ?o1 {bb:InformationManagement } . }
    
    # 4/1 What are hospital information systems?
    SELECT DISTINCT ?o1
    WHERE
      { VALUES ?o1 { bb:HospitalInformationSystem } . }
    
    # 4/2 What are transinstitutional health information systems?
    SELECT DISTINCT ?o1
    WHERE
      { VALUES ?o1 { bb:TransinstitutionalHealthInformationSystem } . }
    
    # 4/4 What are electronic health records?
    SELECT DISTINCT ?o1
    WHERE
      { VALUES ?o1 { bb:ElectronicHealthRecord } . }
    
    # 5/2 What are typical metamodels for modeling various aspects of HIS?
    SELECT DISTINCT ?s1
    WHERE
      { ?s1 rdfs:subClassOf+ bb:Metamodel . }
    
    # 5/3 What is 3LGM2?
    SELECT DISTINCT ?o1
    WHERE
      { VALUES ?o1 { bb:3LMG2 } . }
    
    # 5/4 What are typical reference models for HIS?
    SELECT DISTINCT ?s1
    WHERE
      { ?s1 rdfs:subClassOf+ bb:ReferenceModel . }
    
    # 6/2 What are the main hospital functions?
    SELECT DISTINCT ?s1
    WHERE
      { ?s1 rdfs:subClassOf bb:HospitalFunction . }
    
    # 6/4 What are different architectures of HIS?
    SELECT DISTINCT ?s1
    WHERE
      { ?s1 rdfs:subClassOf+ bb:HisArchitecture . }
    
    # 6.5/1 How can architectures of HIS be categorized?
    SELECT DISTINCT ?s1
    WHERE
      { ?s1 rdfs:subClassOf+ bb:HisArchitecture . }
    
    # 6.6/2 What is meant by the term "infrastructure"?
    SELECT DISTINCT ?o1
    WHERE
      { VALUES ?o1 { bb:HisInfrastructure } . }
    
    # 6.7/2 What is meant by physical integration?
    SELECT DISTINCT ?o1
    WHERE
      { VALUES ?o1 { bb:PhysicalIntegration } . }
    
    # 7/3 Which strategies are appropriate for maintaining electronic health records in a transinstitutional health information system?
    SELECT DISTINCT ?s1
    WHERE
      { ?s1 rdfs:subClassOf bb:EhrStrategy . }
    
    # 8/1 Which facets of quality have to be considered in HIS?
    SELECT DISTINCT ?s1
    WHERE
      { bb:HisQuality meta:entityTypeComponent ?o1 . }
    
    # 8/2 What are the characteristics of the quality of structures in HIS?
    SELECT DISTINCT ?s2
    WHERE
      { bb:QualityOfHISStructures meta:entityTypeComponent ?o1 .
        ?s2 rdfs:subClassOf ?o1 . }
    
    # 8/3 What are the characteristics of  the quality of processes of HIS?
    SELECT DISTINCT ?s1
    WHERE
      { ?s1 rdfs:subClassOf bb:QualityOfHISProcesses . }
    
    # 8/6 How can quality of HIS be evaluated?
    SELECT DISTINCT ?s1
    WHERE
      { ?s1 rdfs:subClassOf+ bb:EvaluationMethod . }
    
    # 8.2/1 What criteria for quality of data exist?
    SELECT DISTINCT ?s1
    WHERE
      { ?s1 rdfs:subClassOf bb:QualityOfData . }
    
    # 8.2/3 What criteria for the overall HIS architecture exist?
    SELECT DISTINCT ?s1
    WHERE
      { ?s1 rdfs:subClassOf bb:QualityOfHISArchitecture . }
    
    # 8.3/1 What are the characteristics of the quality of processes of HIS?
    SELECT DISTINCT ?s1
    WHERE
      { ?s1 rdfs:subClassOf bb:QualityOfHISProcesses . }
    
    # 8.6/1 What are major phases of an IT evaluation study?
    SELECT DISTINCT ?o1
    WHERE
      { bb:ItEvaluationStudyManagementAndExecution meta:functionComponent ?o1 . }
    
    # 8.6/2 What are major IT evaluation methods?
    SELECT DISTINCT ?s1
    WHERE
      { ?s1 rdfs:subClassOf bb:EvaluationMethod . }
    
    # 9/3 What are the tasks and methods for strategic HIS planning?
    SELECT DISTINCT ?o1
    WHERE
      { bb:StrategicHISPlanning meta:functionComponent ?o1 . }
    
    # 9/4 What are the tasks and methods for strategic HIS monitoring?
    SELECT DISTINCT ?o1
    WHERE
      { bb:StrategicHISMonitoring meta:functionComponent ?o1 . }
    
    # 9/5 What are the tasks and methods for strategic HIS directing?
    SELECT DISTINCT ?o1
    WHERE
      { bb:StrategicHISDirecting meta:functionComponent ?o1 . }
    
    # 9.2/2 What are the three main scopes of information management?
    SELECT DISTINCT ?o1
    WHERE
      { bb:InformationManagement meta:functionComponent ?o1 . }
    
    # 9.3.4.2/1 Which organizational units are involved in information management?
    SELECT DISTINCT ?s1
    WHERE
      { ?s1 meta:functionComponent bb:InformationManagement . }
    
    # 9.3.4.2/2 Which boards and persons are involved in information management?
    SELECT DISTINCT ?s1
    WHERE
      { ?s1 meta:isResponsibleForFunction bb:InformationManagement . }
    
    # 9.4/2 What are the typical tasks for strategic HIS planning?
    SELECT DISTINCT ?s1
    WHERE
      { ?s1 rdfs:subClassOf bb:StrategicHISPlanning . }
    
    # 9.5/1 What are the typical tasks of strategic HIS monitoring?
    SELECT DISTINCT ?s1
    WHERE
      { ?s1 rdfs:subClassOf bb:StrategicHISMonitoring . }
    
    # 9.5/2 What are the typical methods of strategic HIS monitoring?
    SELECT DISTINCT ?o1
    WHERE
      { bb:StrategicHISMonitoring meta:uses ?o1 . }
    
    # 9.6/1 What are the typical tasks of strategic HIS directing?
    SELECT DISTINCT ?o1
    WHERE
      { bb:StrategicHISDirecting meta:functionComponent ?o1 . }
    
    # 9.6/2 What are the typical methods of strategic HIS directing?
    SELECT DISTINCT ?o1
    WHERE
      { bb:StrategicHISDirecting meta:uses ?o1 . }
    
    # 10/1 What are health care networks?
    SELECT DISTINCT ?o1
    WHERE
      { VALUES ?o1 {bb:HealthCareNetwork } . }
    
    # 10/2 How can health care networks be described?
    SELECT DISTINCT ?o1
    WHERE
      { VALUES ?o1 {bb:HealthCareNetwork } . }
    
    \end{lstlisting}

\section[Antworten auf die SPARQL-Abfragen]{Antworten auf die \ac{sparql}-Abfragen}\label{sub:antwortentextbuch}

Zuletzt eine Auflistung der Antworten auf die \ac{sparql}-Abfragen aus \cref{sub:sparqltextbuchfragen}.
Diese können über \url{https://www.snik.eu/sparql} ausgegeben werden.

\textbf{Frage \texttt{1/2}:} What are appropriate models for health information systems?

\begin{itemize}
  \item Technical Model (\aurl{bb}{TechnicalModel})
  \item Strategic Alignment Model (\aurl{bb}{StrategicAlignmentModel})
  \item Reference Model (\aurl{bb}{ReferenceModel})
  \item Organizational Model (\aurl{bb}{OrganizationalModel})
  \item OpenEHR Model of Processes (\aurl{bb}{OpenEHRModelOfProcesses})
  \item OpenEHR Model of Content (\aurl{bb}{OpenEHRModelOfContent})
  \item Model of the Planned HIS (\aurl{bb}{ModelOfThePlannedHIS})
  \item Model of the Current HIS (\aurl{bb}{ModelOfTheCurrentHIS})
  \item Information System Model (\aurl{bb}{InformationSystemModel})
  \item Information Processing Model (\aurl{bb}{InformationProcessingModel})
  \item Functional Model (\aurl{bb}{FunctionalModel})
  \item Data Reference Model (\aurl{bb}{DataReferenceModel})
  \item Data Model (\aurl{bb}{DataModel})
  \item Business Reference Model (\aurl{bb}{BusinessReferenceModel})
  \item Business Process Model (\aurl{bb}{BusinessProcessModel})
  \item UML Activity Diagram (\aurl{bb}{UmlActivityDiagram})
  \item Process Chain (\aurl{bb}{ProcessChain})
  \item Petri Net (\aurl{bb}{PetriNet})
  \item Event-Driven Process Chain (\aurl{bb}{EventDrivenProcessChain})
  \item UML Class Diagram (\aurl{bb}{UmlClassDiagram})
  \item Class Diagram (\aurl{bb}{ClassDiagram})
  \item Reference Model for the Domain Layer of Hospital Information Systems (\aurl{bb}{ReferenceModelForTheDomainLayerOfHospitalInformationSystems})
  \item OAIS3 (\aurl{bb}{OAIS3})
  \item ISO/OSI Reference Model (\aurl{bb}{IsoosiReferenceModel})
  \item IHE Integration Profile (\aurl{bb}{IheIntegrationProfile})
  \item HL7 Reference Information Model (\aurl{bb}{HL7ReferenceInformationModel})
  \item Tan’s Critical Success Factor Approach (\aurl{bb}{TansCriticalSuccessFactorApproach})
  \item Component Alignment Model (\aurl{bb}{ComponentAlignmentModelOfMartin})
  \item IHE Patient Demographics Query (\aurl{bb}{IhePatientDemographicsQuery})
  \item Cross-Enterprise Document Sharing (\aurl{bb}{CrossEnterpriseDocumentSharing})
\end{itemize}

\textbf{Frage \texttt{3/3}:} What is information management?

\begin{itemize}
  \item Information Management (\aurl{bb}{InformationManagement})
\end{itemize}

\textbf{Frage \texttt{4/1}:} What are hospital information systems?

\begin{itemize}
  \item Hospital Information System (\aurl{bb}{HospitalInformationSystem})
\end{itemize}

\textbf{Frage \texttt{4/2}:} What are transinstitutional health information systems?

\begin{itemize}
  \item Transinstitutional Health Information System \\
  (\aurl{bb}{TransinstitutionalHealthInformationSystem}) % Sonst kein ordentlihcher Zeilenumbruch
\end{itemize}

\textbf{Frage \texttt{4/4}:} What are electronic health records?

\begin{itemize}
  \item Electronic Health Record (\aurl{bb}{ElectronicHealthRecord})
\end{itemize}

\textbf{Frage \texttt{5/2}:} What are typical metamodels for modeling various aspects of HIS?

\begin{itemize}
  \item Technical Metamodel (\aurl{bb}{TechnicalMetamodel})
  \item Organizational Metamodel (\aurl{bb}{OrganizationalMetamodel})
  \item Information System Metamodel (\aurl{bb}{InformationSystemMetamodel})
  \item Functional Metamodel (\aurl{bb}{FunctionalMetamodel})
  \item Data Metamodel (\aurl{bb}{DataMetamodel})
  \item Business Process Metamodel (\aurl{bb}{BusinessProcessMetamodel})
  \item HL7 Reference Information Model (\aurl{bb}{HL7ReferenceInformationModel})
  \item 3LGM$^{2}$ (\aurl{bb}{3LGM2})
  \item 3LGM$^{2}$-S (\aurl{bb}{3LGM2S})
  \item 3LGM$^{2}$-M (\aurl{bb}{3LGM2M})
  \item 3LGM$^{2}$-B (\aurl{bb}{3LGM2B})
\end{itemize}

\textbf{Frage \texttt{5/3}:} What is 3LGM2?

\begin{itemize}
  \item 3LGM$^{2}$ (\aurl{bb}{3LGM2})
\end{itemize}

\textbf{Frage \texttt{5/4}:} What are typical reference models for HIS?

\begin{itemize}
  \item Reference Model for the Domain Layer of Hospital Information Systems (\aurl{bb}{ReferenceModelForTheDomainLayerOfHospitalInformationSystems})
  \item OAIS3 (\aurl{bb}{OAIS3})
  \item ISO/OSI Reference Model (\aurl{bb}{IsoosiReferenceModel})
  \item IHE Integration Profile (\aurl{bb}{IheIntegrationProfile})
  \item HL7 Reference Information Model (\aurl{bb}{HL7ReferenceInformationModel})
  \item IHE Patient Demographics Query (\aurl{bb}{IhePatientDemographicsQuery})
  \item Cross-Enterprise Document Sharing (\aurl{bb}{CrossEnterpriseDocumentSharing})
\end{itemize}

\textbf{Frage \texttt{6/2}:} What are the main hospital functions?

\begin{itemize}
  \item Administrative Function (\aurl{bb}{Administration})
  \item Research and Education Function (\aurl{bb}{EducationResearch})
  \item Management (\aurl{bb}{Management})
  \item Patient Care(\aurl{bb}{PatientCare})
\end{itemize}

\textbf{Frage \texttt{6/4}:} What are different Architectures of HIS?

\begin{itemize}
  \item Homogeneous Architecture (\aurl{bb}{HomogeneousArchitecture})
  \item Heterogeneous Architecture (\aurl{bb}{HeterogeneousArchitecture})
\end{itemize}

\textbf{Frage \texttt{6.5/1}:} How can architectures of HIS be categorized?

\begin{itemize}
  \item Homogeneous Architecture (\aurl{bb}{HomogeneousArchitecture})
  \item Heterogeneous Architecture (\aurl{bb}{HeterogeneousArchitecture})
\end{itemize}

\textbf{Frage \texttt{6.6/2}:} What is meant by the term \enquote{infrastructure}?

\begin{itemize}
  \item HIS Infrastructure (\aurl{bb}{HisInfrastructure})
\end{itemize}

\textbf{Frage \texttt{6.7/2}:} What is meant by the term physical integration?

\begin{itemize}
  \item Physical Integration (\aurl{bb}{PhysicalIntegration})
\end{itemize}

\textbf{Frage \texttt{7/3}:} Which strategies are appropriate for maintaining electronic health records in a transinstitutional health information system?

\begin{itemize}
  \item The Strategy of Independent Health Banks \\
  (\aurl{bb}{TheStrategyOfIndependentHealthBanks})
  \item The Regional- or National-Centric Strategy \\
  (\aurl{bb}{TheRegionalOrNationalCentricStrategy})
  \item The Provider-Centric Strategy (\aurl{bb}{TheProviderCentricStrategy})
  \item Patient-Centric Strategy (\aurl{bb}{PatientCentricStrategy})
\end{itemize}

\textbf{Frage \texttt{8/1}:} Which facets of quality have to be considered in HIS?

\begin{itemize}
  \item Quality of HIS Outcome (\aurl{bb}{QualityOfHISOutcome})
  \item Quality of HIS Processes (\aurl{bb}{QualityOfHISProcesses})
  \item Quality of HIS Structures (\aurl{bb}{QualityOfHISStructures})
\end{itemize}

\textbf{Frage \texttt{8/2}:} What are the characteristics of the quality of structures in HIS?

\begin{itemize}
  \item Access Integration (\aurl{bb}{AccessIntegration})
  \item Data Integration (\aurl{bb}{DataIntegration})
  \item Functional Coverage of the Application Component \\
  (\aurl{bb}{FunctionalCoverageOfTheApplicationComponent})
  \item Functional Integration (\aurl{bb}{FunctionalIntegration})
  \item ISO 9241-110 User Interface Design Quality \\
  (\aurl{bb}{ISO9241110UserInterfaceDesignQuality})
  \item Performance of Application Components \\
  (\aurl{bb}{PerformanceOfApplicationComponents})
  \item Semantic Integration (\aurl{bb}{SemanticIntegration})
  \item Software Ergonomics (\aurl{bb}{SoftwareErgonomics})
  \item Software Quality (\aurl{bb}{SoftwareQuality})
  \item Stability of Application Components (\aurl{bb}{StabilityOfApplicationComponents})
  \item Accuracy (\aurl{bb}{Accuracy})
  \item Authenticity of Data (\aurl{bb}{AuthenticityOfData})
  \item Availability of Data (\aurl{bb}{AvailabilityOfData})
  \item Completeness (\aurl{bb}{Completeness})
  \item Confidentiality (\aurl{bb}{Confidentiality})
  \item Durability (\aurl{bb}{Durability})
  \item Integrity of Data (\aurl{bb}{IntegrityOfData})
  \item Relevancy (\aurl{bb}{Relevancy})
  \item Reliability of Data (\aurl{bb}{ReliabilityOfData})
  \item Security of Data (\aurl{bb}{SecurityOfData})
  \item Standardization of Data (\aurl{bb}{StandardizationOfData})
  \item Adaptability of the HIS (\aurl{bb}{AdaptabilityOfTheHIS})
  \item Balance of Computer-Based and Non-Computer-Based Tools \\
  (\aurl{bb}{BalanceOfComputerBasedAndNonComputerBasedTools})
  \item Balance of Data Security and Working Processes \\
  (\aurl{bb}{BalanceOfDataSecurityAndWorkingProcesses})
  \item Balance of Documentation Quality and Documentation Efforts \\
  (\aurl{bb}{BalanceOfDocumentationQualityAndDocumentationEfforts})
  \item Balance of Functional Leanness and Functional Redundancy \\
  (\aurl{bb}{BalanceOfFunctionalLeannessAndFunctionalRedundancy})
  \item Balance of Homegeneity and Heterogeneity \\
  (\aurl{bb}{BalanceOfHomegeneityAndHeterogeneity})
  \item Controlled Redundancy of Data (\aurl{bb}{ControlledRedundancyOfData})
  \item Functional Leanness (\aurl{bb}{FunctionalLeanness})
  \item Functional Redundancy (\aurl{bb}{FunctionalRedundancy})
  \item Heterogeneity of the HIS Architecture (\aurl{bb}{HeterogeneityOfTheHISArchitecture})
  \item Homogeneity of the HIS Architecture (\aurl{bb}{HomogeneityOfTheHISArchitecture})
  \item Saturation (\aurl{bb}{Saturation})
  \item Transparency (\aurl{bb}{Transparency})
\end{itemize}

\textbf{Frage \texttt{8/3}:} What are the characteristics of the quality of processes of HIS?

\begin{itemize}
  \item Controlled Transcription of Data (\aurl{bb}{ControlledTranscriptionOfData})
  \item Efficiency of Information Logistics (\aurl{bb}{EfficiencyOfInformationLogistics})
  \item Leanness of Information Processing Tools \\
  (\aurl{bb}{LeannessOfInformationProcessingTools})
  \item Multiple Usability of Data (\aurl{bb}{MultipleUsabilityOfData})
  \item Patient-Centered Information Processing \\
  (\aurl{bb}{PatientCenteredInformationProcessing})
\end{itemize}

\textbf{Frage \texttt{8/6}:} How can quality of HIS be evaluated?

\begin{itemize}
  \item Usability Study (\aurl{bb}{UsabilityStudy})
  \item SWOT Analysis (\aurl{bb}{SwotAnalysis})
  \item Quantitative Evaluation Method (\aurl{bb}{QuantitativeEvaluationMethod})
  \item Qualitative Evaluation Method (\aurl{bb}{QualitativeEvaluationMethod})
  \item Delphi Survey (\aurl{bb}{DelphiSurvey})
  \item Qualitative Observation (\aurl{bb}{QualitativeObservation})
  \item Qualitative Interview (\aurl{bb}{QualitativeInterview})
  \item Qualitative Content Analysis (\aurl{bb}{QualitativeContentAnalysis})
  \item Case Study (\aurl{bb}{CaseStudy})
  \item Utility Analysis (\aurl{bb}{UtilityAnalysis})
  \item User Survey (\aurl{bb}{UserSurvey})
  \item Time Measurement (\aurl{bb}{TimeMeasurement})
  \item Return-on-Investment Study (\aurl{bb}{ReturnOnInvestmentStudy})
  \item Quantitative Questionnaire (\aurl{bb}{QuantitativeQuestionnaire})
  \item Patient Satisfaction Survey (\aurl{bb}{PatientSatisfactionSurvey})
  \item Event Counting Study (\aurl{bb}{EventCountingStudy})
  \item Effectiveness Study (\aurl{bb}{EffectivenessStudy})
  \item Cost-Effectiveness Analysis (\aurl{bb}{CostEffectivenessAnalysis})
  \item Cost–Benefit Analysis (\aurl{bb}{CostBenefitAnalysis})
  \item Consensus Method (\aurl{bb}{ConsensusMethod})
  \item Work Sampling (\aurl{bb}{WorkSampling})
  \item Time-Motion Analysis (\aurl{bb}{TimeMotionAnalysis})
  \item Unstructured Interview (\aurl{bb}{UnstructuredInterview})
  \item Semistructured Interview (\aurl{bb}{SemistructuredInterview})
\end{itemize}

\textbf{Frage \texttt{8.2/1}:} What criteria for quality of data exist?

\begin{itemize}
  \item Accuracy (\aurl{bb}{Accuracy})
  \item Authenticity of Data (\aurl{bb}{AuthenticityOfData})
  \item Availability of Data (\aurl{bb}{AvailabilityOfData})
  \item Completeness (\aurl{bb}{Completeness})
  \item Confidentiality (\aurl{bb}{Confidentiality})
  \item Durability (\aurl{bb}{Durability})
  \item Integrity of Data (\aurl{bb}{IntegrityOfData})
  \item Relevancy (\aurl{bb}{Relevancy})
  \item Reliability of Data (\aurl{bb}{ReliabilityOfData})
  \item Security of Data (\aurl{bb}{SecurityOfData})
  \item Standardization of Data (\aurl{bb}{StandardizationOfData})
\end{itemize}

\textbf{Frage \texttt{8.2/3}:} What criteria for the overall HIS architecture exist?

\begin{itemize}
  \item Adaptability of the HIS (\aurl{bb}{AdaptabilityOfTheHIS})
  \item Balance of Computer-Based and Non-Computer-Based Tools \\
  (\aurl{bb}{BalanceOfComputerBasedAndNonComputerBasedTools})
  \item Balance of Data Security and Working Processes \\
  (\aurl{bb}{BalanceOfDataSecurityAndWorkingProcesses})
  \item Balance of Documentation Quality and Documentation Efforts \\(\aurl{bb}{BalanceOfDocumentationQualityAndDocumentationEfforts})
  \item Balance of Functional Leanness and Functional Redundancy \\
  (\aurl{bb}{BalanceOfFunctionalLeannessAndFunctionalRedundancy})
  \item Balance of Homegeneity and Heterogeneity \\
  (\aurl{bb}{BalanceOfHomegeneityAndHeterogeneity})
  \item Controlled Redundancy of Data (\aurl{bb}{ControlledRedundancyOfData})
  \item Functional Leanness (\aurl{bb}{FunctionalLeanness})
  \item Functional Redundancy (\aurl{bb}{FunctionalRedundancy})
  \item Heterogeneity of the HIS Architecture (\aurl{bb}{HeterogeneityOfTheHISArchitecture})
  \item Homogeneity of the HIS Architecture (\aurl{bb}{HomogeneityOfTheHISArchitecture})
  \item Saturation (\aurl{bb}{Saturation})
  \item Transparency (\aurl{bb}{Transparency})
\end{itemize}

\textbf{Frage \texttt{8.3/1}:} What are the characteristics of the quality of processes of HIS?

\begin{itemize}
  \item Controlled Transcription of Data (\aurl{bb}{ControlledTranscriptionOfData})
  \item Efficiency of Information Logistics (\aurl{bb}{EfficiencyOfInformationLogistics})
  \item Leanness of Information Processing Tools \\
  (\aurl{bb}{LeannessOfInformationProcessingTools})
  \item Multiple Usability of Data (\aurl{bb}{MultipleUsabilityOfData})
  \item Patient-Centered Information Processing \\
  (\aurl{bb}{PatientCenteredInformationProcessing})
\end{itemize}

\textbf{Frage \texttt{8.6/1}:} What are major phases of an IT evaluation study?

\begin{itemize}
  \item Execution of an IT Evaluation Study (\aurl{bb}{ExecutionOfAnITEvaluationStudy})
  \item First Study Design (\aurl{bb}{FirstStudyDesign})
  \item Operationalization of Methods and Detailed Study Plan \\
  (\aurl{bb}{OperationalizationOfMethodsAndDetailedStudyPlan})
  \item Report and Publication of Study (\aurl{bb}{ReportAndPublicationOfStudy})
  \item Study Exploration (\aurl{bb}{StudyExploration})
\end{itemize}

\textbf{Frage \texttt{8.6/2}:} What are major IT evaluation methods?

\begin{itemize}
  \item Delphi Survey (\aurl{bb}{DelphiSurvey})
  \item Qualitative Evaluation Method (\aurl{bb}{QualitativeEvaluationMethod})
  \item Quantitative Evaluation Method (\aurl{bb}{QuantitativeEvaluationMethod})
  \item SWOT Analysis (\aurl{bb}{SwotAnalysis})
  \item Usability Study (\aurl{bb}{UsabilityStudy})
\end{itemize}

\textbf{Frage \texttt{9/3}:} What are the tasks and methods for strategic HIS planning?

\begin{itemize}
  \item Long-Term HIS Planning(\aurl{bb}{LongTermHISPlanning})
  \item Portfolio Management(\aurl{bb}{PortfolioManagement})
  \item Short-Term HIS Planning (\aurl{bb}{ShortTermHISPlanning})
\end{itemize}

\textbf{Frage \texttt{9/4}:} What are the tasks and methods for strategic HIS monitoring?

\begin{itemize}
  \item Ad Hoc Monitoring (\aurl{bb}{AdHocMonitoring})
  \item HIS Certification (\aurl{bb}{HisCertification})
  \item Permanent Monitoring (\aurl{bb}{PermanentMonitoring})
  \item Strategic Information Management (\aurl{bb}{StrategicInformationManagement})
\end{itemize}

\textbf{Frage \texttt{9/5}:} What are the tasks and methods for strategic HIS directing?

\begin{itemize}
  \item Adoption of Project Result (\aurl{bb}{AdoptionOfProjectResult})
  \item Monitoring the Project Progress (\aurl{bb}{MonitoringTheProjectProgress})
  \item Project Initiation (\aurl{bb}{ProjectInitiation})
  \item Project Resource Allocation (\aurl{bb}{ProjectResourceAllocation})
  \item Project Time Allocation (\aurl{bb}{ProjectTimeAllocation})
  \item Strategic Information Management (\aurl{bb}{StrategicInformationManagement})
\end{itemize}

\textbf{Frage \texttt{9.2/2}:} What are the three main scopes of information management?

\begin{itemize}
  \item IT Service Management (\aurl{bb}{ItServiceManagement})
  \item Modeling Information Systems (\aurl{bb}{ModelingInformationSystems})
  \item Operational Information Management (\aurl{bb}{OperationalInformationManagement})
  \item Strategic Information Management (\aurl{bb}{StrategicInformationManagement})
  \item Tactical Information Management (\aurl{bb}{TacticalInformationManagement})
\end{itemize}

\textbf{Frage \texttt{9.3.4.2/1}:} Which organizational units are involved in information management?

\begin{itemize}
  \item Hospital Administration (\aurl{bb}{HospitalAdministration})
\end{itemize}

\textbf{Frage \texttt{9.3.4.2/2}:} Which boards and persons are involved in information management?

\begin{itemize}
  \item Chief Information Officer (\aurl{bb}{ChiefInformationOfficer})
\end{itemize}

\textbf{Frage \texttt{9.4/2}:} What are the typical tasks for strategic HIS planning?

\begin{itemize}
  \item HIS Budget Planning (\aurl{bb}{HisBudgetPlanning})
  \item IT Investment Planning (\aurl{bb}{ItInvestmentPlanning})
  \item Long-Term HIS Planning (\aurl{bb}{LongTermHISPlanning})
  \item Short-Term HIS Planning (\aurl{bb}{ShortTermHISPlanning})
  \item Strategic Alignment (\aurl{bb}{StrategicAlignment})
\end{itemize}

\textbf{Frage \texttt{9.5/1}:} What are the typical tasks of strategic HIS monitoring?

\begin{itemize}
  \item Ad Hoc Monitoring (\aurl{bb}{AdHocMonitoring})
  \item Continuous HIS Auditing (\aurl{bb}{ContinuousHISAuditing})
  \item Continuous Quality Improvement Process \\
  (\aurl{bb}{ContinuousQualityImprovementProcess})
  \item IT Investment Justification (\aurl{bb}{ItInvestmentJustification})
  \item Report to the Hospital’s Board of Directors \\
  (\aurl{bb}{ReportToTheHospitalsBoardOfDirectors})
  \item Reporting to the CEO (\aurl{bb}{ReportingToTheCEO})
  \item Structural Quality Assessment (\aurl{bb}{StructuralQualityAssessment})
\end{itemize}

\textbf{Frage \texttt{9.5/2}:} What are the typical methods of strategic HIS monitoring?

\begin{itemize}
  \item Functional Redundancy Rate (\aurl{bb}{FunctionalRedundancyRate})
  \item HIS Quality (\aurl{bb}{HisQuality})
  \item Key Performance Indicator (\aurl{bb}{KeyPerformanceIndicator})
  \item Strategic Information Management Plan (\aurl{bb}{StrategicInformationManagementPlan})
\end{itemize}

\textbf{Frage \texttt{9.6/1}:} What are the typical tasks of strategic HIS directing?

\begin{itemize}
  \item Adoption of Project Result (\aurl{bb}{AdoptionOfProjectResult})
  \item Monitoring the Project Progress (\aurl{bb}{MonitoringTheProjectProgress})
  \item Project Initiation (\aurl{bb}{ProjectInitiation})
  \item Project Resource Allocation (\aurl{bb}{ProjectResourceAllocation})
  \item Project Time Allocation (\aurl{bb}{ProjectTimeAllocation})
  \item Strategic Information Management (\aurl{bb}{StrategicInformationManagement})
\end{itemize}

\textbf{Frage \texttt{9.6/2}:} What are the typical methods of strategic HIS directing?

\begin{itemize}
  \item HIS Architecture (\aurl{bb}{HisArchitecture})
  \item Information Management Organizational Structure \\
  (\aurl{bb}{InformationManagementOrganizationalStructure})
  \item Migration Path (\aurl{bb}{MigrationPath})
  \item Project Portfolio (\aurl{bb}{ProjectPortfolio})
  \item Strategic HIS Monitoring Result (\aurl{bb}{StrategicHISMonitoringResult})
  \item Strategic Information Management Plan \\
  (\aurl{bb}{StrategicInformationManagementPlan})
\end{itemize}

\textbf{Frage \texttt{10/1}:} What are health care networks?

\begin{itemize}
  \item Health Care Network (\aurl{bb}{HealthCareNetwork})
\end{itemize}

\textbf{Frage \texttt{10/2}:} How can health care networks be described?

\begin{itemize}
  \item Health Care Network (\aurl{bb}{HealthCareNetwork})
\end{itemize}
