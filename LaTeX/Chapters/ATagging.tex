\chapter{Syntaktisches und POS-Tagging}\label{ch:tagging}

Sowohl \ac{pos}- als auch syntaktisches Tagging, also das Tagging von entweder einzelnen Wörtern oder Satzbauteilen, geschieht nach verschiedenen Standards.
In dieser Arbeit wird, wenn es denn geschieht, die Penn Treebank-Notation verwendet.
Diese wird im Folgenden so aufgeschlüsselt, wie es hier benötigt wird.
Es wurden jedoch einzelne Zeichen o.ä. ausgelassen, wie etwa einzelne Tags für verschiedene Anführungszeichen beim \ac{pos}-Tagging.

\begin{longtable}{r l}
  \toprule
  Abkürzung & Bedeutung \\
  \midrule \\
  \endhead
  ADJP & Adjective phrase \\
  ADVP & Adverb phrase \\
  NP & Noun phrase \\
  PP & Prepositional phrase \\
  S & Simple declarative clause \\
  SBAR & Subordinate clause \\
  SBARQ & Direct question introduced by wh-element \\
  SINV & Declarative sentence with subject-aux inversion \\
  SQ & Yes/no questions and subconstituent of SBARQ excluding wh-element \\
  VP & Verb phrase \\
  WHADVP & Wh-adverb phrase \\
  WHNP & Wh-noun phrase \\
  WHPP & Wh-prepositional phrase \\
  X & Constituent of unknown or uncertain category \\
  \bottomrule
  \caption[Penn Treebank-Notation syntaktischer Tags]{Penn Treebank-Notation syntaktischer Tags nach \citet{penntreebankpos}}
  \label{tab:penntreebanksynt}
\end{longtable}

\begin{longtable}{r c l}
  \toprule
  Abkürzung & Bedeutung & Beispiel \\
  \midrule \\
  \endhead
  CC & Coordinating conjunction & and, or \\
  CD & Cardinal Number & one, 5 \\
  DT & Determiner & both, some \\
  FW & Foreign word & oui, si \\
  IN & Preposition/subordinating conjunction & upon, if \\
  JJ & adjective & complex, fourty-second \\
  JJR & Adjective, Compartive & simpler, healthier \\
  JJS & Adjective, Superlative &  simplest, healthiest \\
  LS & List item marker & -, 1. \\
  MD & Modal & could, can't \\
  NN & Noun, singular or mass & hospital, knowledge \\
  NNS & Noun, plural & hospitals, systems \\
  NNP & Proper noun, singular & Leipzig, Mustermann \\
  NNPS & Proper noun, plural & Americas, Joneses \\
  PDT & Predeterminer & all, both \\
  POS & Possessive ending & 's \\
  PRP & Personal pronoun & us, it \\
  PP & Possessive pronoun & our, theirs \\
  RB & Adverb & definitely, very \\
  RBR & Adverb, comparative & greater, more \\
  RBS & Adverb, superlative & greatest, most \\
  RP & Particle & up, away \\
  SYM & Symbol & $\subseteq$, § \\
  TO & Infinitival to & to \\
  UH & Interjection & uh, wow \\
  VB & Verb, base form & learn, stay \\
  VBD & Verb, past tense & learned, stayed \\
  VBG & Verb, gerund/present participle & learning, staying \\
  VBN & Verb, past participle & learned, stayed \\
  VBP & Verb, non-3rd person singular present & learn, stay \\
  VBZ & Verb, 3rd person singular present & learns, stays \\
  WDT & Wh-determiner & whichever, what \\
  WP\$ & Wh-pronoun & which, whom \\
  WP\$ & Possessive wh-pronoun & whose \\
  WRB & Wh-adverb & how, why \\
  \bottomrule
  \caption[Penn Treebank-Notation von POS-Tags]{Penn Treebank-Notation von \acs{pos}-Tags nach \citet{penntreebankpos}}
  \label{tab:penntreebankpos}
\end{longtable}
