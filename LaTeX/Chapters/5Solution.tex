%*****************************************
\chapter{Ausführung der Lösung}\label{ch:solution}
%*****************************************

\section{Auswahl eines Kandidaten}

Die Auswahl eines oder mehrerer Kandidaten gestaltete sich aufgrund von nicht instand gehaltener Programme
und der Spezialisierung von vielen Systemen auf DBpedia oder ähnlichen Wissensbasen sehr schwierig.
Bei gAnswer2 war es beispielsweise nicht möglich, überhaupt ein Programm, was man hätte ausprobieren können, zu finden,
bei gAnswer war die Vorbereitung der Daten sehr aufwändig und die gegebenen Werkzeuge kaputt.
DeepPavlov war am Anfang sehr vielversprechend, besonders aufgrund des modularen Aufbaus.
Jedoch gab es keine Möglichkeit, andere Daten als Wikidata zu verwenden, weshalb es für die Verwendung mit \ac{snik} nicht verwendbar ist.
In der Dokumentation\footnote{\url{http://docs.deeppavlov.ai/en/master/features/models/kbqa.html}, abgerufen am 9. Mai 2022} wird allerdings geschrieben,
dass in der Zukunft auch die Verwendung anderer Quellen möglich sein sollen.
All diese und andere Probleme beschrieb bereits \citet{diefenbachkbqa}.
Er nannte speziell die Probleme der
\begin{itemize}
  \item Mehrsprachigkeit,
  \item Portabilität,
  \item Skalierbarkeit,
  \item Robustheit,
  \item Fähigkeit, über mehrere Dateien zu suchen und
  \item Präsentation der Antwort.
\end{itemize}
Hier waren vor allem die Portabilität und Robustheit, aber auch teilweise die Präsentation der Antwort problematisch.

\subsection{TeBaQA}

TeBaQA war anfangs sehr vielversprechend, mehr noch als DeepPavlov.
Es sollte die Verwendung von eigenen \ac{rdf}-Tripeln ermöglichen, indem man die \texttt{indexing.properties}-Datei so verändert, dass die eigene Ontologie verwendet wird.
Es gab zwei Ordner, einen namens \texttt{ontology} und einen namens \texttt{data}.
Da \ac{snik}, wie in \cref{sec:snik} schon erwähnt, eine Ontologien aus Ontologien ist, kam in den \texttt{ontology}-Ordner nur die \texttt{meta.ttl}, wo die Properties gespeichert sind.
Die restlichen Tripel kamen alle in den \texttt{data}-Ordner, da es TeBaQA auch möglich sein sollte, mehrere Dateien als Quelle zu verwenden.
Die auf \texttt{.flag} endenden Variablen wurden auf \texttt{true} gesetzt.
Andere Dateien wurden nicht verändert.

Leider warf das System immer wieder Fehler bezüglich der ElasticSearch-Konfiguration, weshalb wir ein Docker-Image,
also quasi eine auf Anwendungsebene virtuelle Umgebung, mit ElasticSearch 6.6.1, TeBaQA und der \ac{snik}-Ontolgie erstellt.
Diese war zusätzlich noch ein weiteres Mal darin, da es Probleme mit der Reihenfolge des Ladens der Daten gab.
Letztendlich konnten wir aber nicht alle Probleme und Fehler beheben, da es immer wieder zu Laufzeitfehlern kam.

\subsection{QAnswer KG}

QAnswer KG präsentierte sich als einfache Methode, mithilfe von eigenen Daten Question Answering zu betreiben.
Es funktioniert, indem man auf der Website\footnote{\url{https://qanswer-frontend.univ-st-etienne.fr/}} einen Account erstellt und in diesem seine Daten hochlädt.
Auf der Startseite ist bereits ein Beispiel mit Wikidata vorhanden.

\subsubsection{Anfängliche Konfiguration und Probleme}

Beim Einrichten der Umgebung wurde die Sprache auf Englisch gestellt, da in dieser auch die Fragen beantwortet werden sollen.
QAnswer KG versucht, sich auf eine Antwort zu konzentrieren und stellt nur die als am besten bewertete \ac{sparql}-Abfrage dar.
Hier wird das Problem der Ambiguität deutlich, da bei der Frage \enquote{What is the chief information officer responsible for?} sowohl
\aurl{meta}{isResponsibleForEntityType}, \aurl{meta}{isResponsibleForFunction} und \aurl{meta}{isResponsibleForRole} gemeint sein könnten.
Der Ansatz für die Lösung dises Problems war es, im Training in den als Lösung markierten \ac{sparql}-Abfragen \texttt{FILTER} zu verwenden,
mit denen man zum Beispiel das Prädikat unterschiedliche Ressourcen sein lassen kann: \texttt{FILTER (?p1 = :x || ?p1 = :y || ?p1 = :z)}.

Praktisch für die Lokalisierung der Fehler war die Funktion, sich alle generierten Anfragen anzeigen zu lassen.
Somit konnte erahnt werden, warum es das macht, was es macht.

Wenn keine Lösung für die Frage gefunden wurde, gab es als Ausgabe meist die Ressource selbst, also zum Beispiel \ac{bb}{ChiefInformationOfficer}.
Dies kann zwar nützlich, aber auch verwirrend sein und sollte mithilfe von Training verhindert werden.
Die Präzision der Ergebnisse ohne jegliches fine-tuning ist aber trotzdem erstaunlich.

Mit der Konfiguration an sich konnten auch schon viele Fehler behoben werden.
So gibt es eine Liste von \enquote{stop words}, welche nicht betrachtet werden.
Unter diesen sind häufig verwendete Präpositionen, Konjunktionen, Verben oder Füllwörter wie \enquote{and} oder \enquote{many}.
Diese sollen verhindern, dass falsche Ressourcen gefunden werden.
Hier musste diese Liste allerdings so modifiziert werden, dass das Wort \texttt{for} nicht mehr darin vorkommt, denn ohne es können Prädikate wie \enquote{responsible for} schwer erkannt werden,
besonders, da die beiden Wörter oft getrennt im Satz vorkommen.
Somit verblieben dann folgende Wörter in der Liste der Stopp-Wörter:
\texttt{a, about, all, an, and, are, as, at, be, by, can, define, describe, did, do, does, from, give, goes, had, has, have, here, how, in, into, is, its, list, many, most, my, no, of, on, or, s, show, some, something, such, tell, the, their, these, they, this, to, using, was, were, what, which, will, with, yes}.

Aus den \enquote{hidden properties}, welche Properties, welche nie betrachtet werden sollten, enthielten, wurde \texttt{http://www.w3.org/1999/02/22-rdf-syntax-ns\#type} entfernt.
Bei dem Versuch, \texttt{http://www.w3.org/2004/02/skos/core\#definition} als Alias für Labels einzufügen, gab es einen Fehler 500.

\subsubsection{Training}

\begin{itemize}
  \item QAnswer KG
  \begin{itemize}
    \item nur eine Ergebnisquery anstatt potenziell mehrerer (\aurl{meta}{isResponsibleForFunction}, \aurl{meta}{isResponsibleForEntityType})
    \item erkennt Prädikate auch, wenn diese nicht beieinander stehen (\enquote{Wofür ist der Leiter des Informationsmanagements verantwortlich?})
    \begin{itemize}
      \item Hierzu müsste FILTER genutzt werden (?p1 = :x || ?p1 = :y || ?p1 = :z)
      \item Training?
    \end{itemize}
    \item wenn keine Lösung gefunden, wird oft Ressource, die im Subjekt steht, ausgegeben
    \item \url{https://doc.qanswer.eu/Tutorial-UI/doc1.6}
    \begin{itemize}
      \item http://www.w3.org/2004/02/skos/core\#definition für Definition ==> Fehler 500
    \end{itemize}
    \item stop words
    \begin{itemize}
      \item rausnehmen: for
      \item verbleibend: \texttt{a, about, all, an, and, are, as, at, be, by, can, define, describe, did, do, does, from, give, goes, had, has, have, here, how, in, into, is, its, list, many, most, my, no, of, on, or, s, show, some, something, such, tell, the, their, these, they, this, to, using, was, were, what, which, will, with, yes}
    \end{itemize}
    \item hidden properties
    \begin{itemize}
      \item http://www.w3.org/1999/02/22-rdf-syntax-ns\#type entfernen
    \end{itemize}
    \item bevor \enquote{for} hersausgenommen wurde, hat es \enquote{responsible for} selten verstanden, jetzt fast immer als Prädikat
  \end{itemize}
\end{itemize}
