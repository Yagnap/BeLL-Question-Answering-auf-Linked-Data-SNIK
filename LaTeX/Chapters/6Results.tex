%*****************************************
\chapter{Ergebnisse}\label{ch:results}
%*****************************************
Es wurde ein Benchmark aus 36 Lehrbuchfragen, von denen jeweils 18 zufällig der Test- und Trainingsgruppe zugeordnet wurden, erstellt und auf QAnswer verwendet.
Bei diesem gibt es allerdings noch teilweise Probleme bei der Richtigkeit weniger Antworten, welche aber die \ac{sparql}-Abfragen nicht wesentlich beeinflussen sollten.
Deshalb ist er noch unaussagekräftig.
Ein weiterer Benchmark aus 997 automatisch generierten, sehr einfachen Fragen wurde automatisch erstellt, 100 von den Fragen wurden dem Testset, 897 dem Trainingsset zugeordnet.
Die Teilontologie aus \texttt{bb} wurde mit einem Question Answering-System, QAnswer KB, genutzt und mithilfe des Benchmarks aus Lehrbuchfragen trainiert und evaluiert,
wobei die trainierte Version deutlich schlechter und fast schon unverwendbar war, die untrainierte Version hingegen relativ gute Ergebnisse geliefert hat.
