%************************************************
\chapter{Einleitung}\label{ch:introduction}
%************************************************

\section{Gegenstand}
\begin{flushright}{\slshape
\enquote{Die Medizinische Informatik ist die Wissenschaft der systematischen Erschließung, Verwaltung, Aufbewahrung, Verarbeitung und Bereitstellung von Daten, Informationen und Wissen in der Medizin und im Gesundheitswesen.
Sie ist von dem Streben geleitet, damit zur Gestaltung der bestmöglichen Gesundheitsversorgung beizutragen.}\footnote{\url{https://www.gmds.de/de/aktivitaeten/medizinische-informatik/}, abgerufen am 11. Januar 2022}}
\end{flushright}
\acs{snik} ist ein Projekt zum Informationsmanagement im Gesundheitswesen.
Es fasst Wissen aus drei Lehrbüchern, welches es in semantischer Weise modelliert und publiziert.
In der Ontologie (siehe \cref{sub:ontology}) sind Zusammenhänge zwischen verschiedenen Informationen vorhanden,
welche auf der öffentlich erreichbaren Visualisierung in Form eines Graphen dargestellt sind.

Diese Daten liegen strukturiert, aber für Menschen nicht leicht lesbar vor.
Daher braucht es andere Methoden, durch sie zu navigieren.
Dazu gibt es verschiedene Lösungen:
\begin{enumerate}
	\item das Verwenden eines \acs{rdf}-Browsers \citep[S.~30]{linkeddatavisualisation}, der \acs{rdf}-Informationen einzeln darstellt,
	\item die Graphvisualisierung \citep[S.~32]{linkeddatavisualisation} oder
	\item \acs{sparql}, eine Abfragesprache für \acs{rdf}.
\end{enumerate}

\section{Problemstellung}\label{sec:problemstellung}

Momentan müssen Studierende der Medizininformatik, die nach Wissen suchen, auf eine der drei oben genannten Optionen zurückgreifen.
Jede dieser Möglichkeiten hat jedoch große Nachteile.
Der \acs*{rdf}-Browser gibt nur ein sehr beschränktes Ergebnis aus, und serialisiertes \acs*{rdf} selbst zu lesen ist schwer und unpraktisch.
Die Graphvisualisierung kann im Zweifel unübersichtlich oder überwältigend sein, da es schwer sein kann, überhaupt die Ressource zu finden, zu der man eine Frage hat, und dann zur Antwort zu navigieren.
Im Fall von \acs*{sparql} gibt es einen erheblichen Zeitaufwand für die Studierenden, da sie sich dort erst in die Syntax der Abfragesprache und das Vokabular des Fachbereichs einarbeiten müssen.

Daraus ergibt sich das Problem, dass keine der momentan existierenden Lösungen intuitiv genug funktioniert, als dass es nahezu keine Einarbeitungszeit gibt.
Die existierenden Lösungen liefern zudem nicht übersichtlich ausreichend Informationen, ihrer Expressivität sind demnach deutliche Grenzen gesetzt.

Obwohl ein Ansatz für die Lösung dieses Problems besteht, \acl*{qa}, wirft dieser direkt ein neues Problem für die entwickelnden Personen auf.
Die Implementierung eines \acl*{qa}-Systems mit adäquater Qualität der Antworten ist wesentlich aufwändiger~\citep[S.~3]{qanswer}, als es in einem angemessenem Zeitraum bei stark beschränkten Mitteln möglich ist.

\begin{itemize}
	\item Problem P1: Für die Implementierung von \acl{qa} benötigter Aufwand
	\item Problem P2: Fehlende Intuitivität und Expressivität existierender Lösungen
\end{itemize}

\section{Motivation}

Das Wissen zum Informationsmanagement im Krankenhaus ist komplex und oft nur schwer greifbar.
Es liegt in Form von Lehrbüchern, aber auch in \acs{snik} vor.

Studierende haben selten Zeit, sich ganze Kapitel oder gar Bücher durchzulesen, verfügen jedoch auch nicht über die Kenntnisse \acs*{snik} effektiv zu verwenden.
Als Folge müssen sie bei Fragen oft ihren Professor oder andere Studierende hinzuziehen.
Es wäre ungemein einfacher, wenn sie das strukturierte Wissen in natürlicher Sprache abfragen könnten.
\acl*{qa}-Systeme sind im Idealfall 24 Stunden am Tag und 7 Tage die Woche leicht erreichbar und können sofort antworten. 
Besonders in Zeiten der Covid-19-Pandemie, in denen direkte Kontakte mit Studierenden und Professoren oft eingeschränkt werden müssen und es digitale Veranstaltungen ohne örtliche Präsenz gibt, ist solch ein Werkzeug sehr hilfreich.

\section{Zielsetzung}\label{sec:zielsetzung}

Im Folgenden werden beiden Problemen je ein Ziel zugeordnet.

Das Problem P1 des Aufwandes der Implementierung eines eigenen Systems kann durch das Verwenden eines bereits existierenden Systems umgangen werden.
Dieses muss das Wissen \acs{snik}s nutzen, um Fragen zu beantworten.

Das Ziel unterstützt die Lösung des Problems P2 der mangelhaften anderen Möglichkeiten zur Wissensabfrage oder -aneignung in der Medizininformatik.
Das zweite Ziel stellt demnach das Erreichen eines semantischen \acl*{qa}-Systems dar, welches Fragen mit hoher Genauigkeit beantworten kann.
Bei \acl*{qa} wird eine Frage als Satz in natürlicher Sprache eingegeben, dessen Bedeutung das System versteht und auf Grundlage dessen eine Antwort ausgibt \citep{qadefinition}.

\begin{itemize}
	\item Ziel Z1: Benchmark für \acl*{sqa} für \acs{snik}
	\item Ziel Z2: \acl*{sqa}-System, welches Fragen zu \acs{snik} mit hoher Qualität beantwortet.
\end{itemize}
\section{Aufgabenstellung}

Den Zielen sind hier jeweils Aufgaben zugeordnet.

\begin{itemize}

	\item Aufgaben zu Ziel Z1:
	\begin{itemize}
		\item Aufgabe A1.1: Sammlung von typischen Benutzerfragen an \acs{snik}
		\item Aufgabe A1.2: Beantwortung dieser Fragen
		\item Aufgabe A1.3: Methodik entwickeln, um ein System anhand dieser Fragen und Antworten zu bewerten (Benchmark)
	\end{itemize}
	\item Aufgaben zu Ziel Z2:
	\begin{itemize}
		\item Aufgabe A2.1: Recherche existierender \acl*{sqa} Ansätze und Systeme
		\item Aufgabe A2.2: Auswahl eines geeigneten Systems
		\item Aufgabe A2.3: Evaluation dieses Systems am Benchmark
		\item Aufgabe A2.4: Diskussion der Ergebnisse
	\end{itemize}
\end{itemize}

\section{Aufbau der Arbeit}

Einleitend werden in Kapitel 1 die Problemstellung, Ziele und daraus folgende Aufgaben beschrieben.
Danach werden in Kapitel 2 grundlegende Begriffe und für die Arbeit relevante Themengebiete geklärt.
In Kapitel 3 wird der aktuelle Stand der Forschung beschrieben, besonders in Hinsicht auf verschiedene \acl*{qa}-Systeme.
Kapitel 4 beinhaltet die Lösungsansätze für die oben benannten Aufgaben, Kapitel 5 die Ausführung dieser und Kapitel 6 Ergebnisse.
Kapitel 7 ist die kritische Auseinandersetzung mit den Ergebnissen in der Diskussion und ein Ausblick auf mögliche zukünftige Arbeit.
