%************************************************
\chapter{Einleitung}\label{ch:introduction}
%************************************************

\section{Gegenstand}
\begin{flushright}{\slshape
\enquote{Die Medizinische Informatik ist die Wissenschaft der systematischen Erschließung, Verwaltung, Aufbewahrung, Verarbeitung und Bereitstellung von Daten, Informationen und Wissen in der Medizin und im Gesundheitswesen.
Sie ist von dem Streben geleitet, damit zur Gestaltung der bestmöglichen Gesundheitsversorgung beizutragen.\footnote{\url{https://www.gmds.de/de/aktivitaeten/medizinische-informatik/}, abgerufen am 11. Januar 2022}
}}
\end{flushright}
\acs{snik} ist ein Projekt zum Informationsmanagement im Gesundheitswesen.
Es fasst Wissen aus drei Lehrbüchern, welches es in semantischer Weise modelliert und publiziert.
In der Ontologie (siehe \cref{sub:ontology}) sind Zusammenhänge zwischen verschiedenen Informationen vorhanden,
welche auf der öffentlich erreichbaren Visualisierung in Form eines Graphen dargestellt sind.

Diese Daten liegen strukturiert, aber für Menschen nicht leicht lesbar vor.
Daher braucht es andere Methoden, durch sie zu navigieren.
Dazu gibt es verschiedene Lösungen:
\begin{enumerate}
	\item das Verwenden eines \acs{rdf}-Browsers \citep[S.~30]{linkeddatavisualisation}, der \acs{rdf}-Informationen einzeln darstellt,
	\item die Graphvisualisierung \citep[S.~32]{linkeddatavisualisation} oder
	\item \acs{sparql}, eine Abfragesprache für \acs{rdf}.
\end{enumerate}

\section{Problemstellung}\label{sec:problemstellung}

Momentan müssen Studierende der Medizininformatik, die nach Wissen suchen wollen, eine der drei oben genannten Optionen verwenden.
Alle dieser Möglichkeiten haben jedoch große Nachteile.
Der RDF-Browser gibt nur ein sehr beschränktes Ergebnis aus, und serialisiertes RDF selbst zu lesen ist schwer und unpraktisch.
Im Fall von SPARQL nimmt es den Medizininformatik-Studierenden wertvolle Zeit, da sie sich dort erst in die Syntax der Abfragesprache und das Vokabular des Fachbereichs einarbeiten müssen.

Ein Ansatz für die Lösung dieses Problems ist Question Answering.
\todo{Das stimmt zwar alles, gehört aber nicht hierher zu der Problemstellung sondern zu den Zielen und Aufgaben.}
Dabei wird eine Frage als ganzer Satz eingegeben, dessen Bedeutung das System versteht und auf Grundlage dessen eine Antwort ausgibt \citep{qadefinition}.
Die Implementierung eines Question Answering-Systems mit adäquater Qualität der Antworten ist wesentlich aufwändiger~\citep[S.~3]{qanswer}, als es im Rahmen einer BeLL möglich ist.
\todo{Achtung: Die Ebene der Problemstellung soll aus Sicht des zukünftigen Nutzers geschrieben sein. Lies dir bitte noch mal das Kapitel 1.2 aus der Vorlage durch \url{https://github.com/IMISE/imise-classicthesis/files/6785593/thesis.pdf}}.
\begin{itemize}
	\item Problem P1: Für die Implementierung von Question Answering benötigter Aufwand
	\item Problem P2: Fehlende Intuitivität und Expressivität existierender Lösungen
\end{itemize}

\section{Motivation}

Das Wissen zum Informationsmanagement im Krankenhaus ist komplex und oft nur schwer greifbar.
Es liegt in Form von Lehrbüchern, aber eben auch in \acs{snik} vor.
Studenten, die unter Zeitstress leiden und sich nicht immer ganze Bücher oder Kapitel durchlesen können, aber auch nicht über die Kenntnisse verfügen, \acs{snik} effektiv zu verwenden,
müssen oft ihren Professor oder andere Studierende hinzuziehen, wenn sie eine Frage haben.
Es wäre viel einfacher, wenn sie das strukturierte Wissen in natürlicher Sprache abfragen könnten.
Question Answering-Systeme sind im Idealfall aber 24 Stunden am Tag und 7 Tage die Woche nur ein paar Klicks entfernt und können sofort antworten. 
Besonders in Zeiten der Covid-19-Pandemie, wo die Kontakte oft eingeschränkt werden müssen und es häufiger digitale Veranstaltungen ohne örtliche Präsenz gibt, ist solch ein Werkzeug sehr hilfreich.

\section{Zielsetzung}\label{sec:zielsetzung}

Im Folgenden werden beiden Problemen je ein Ziel zugeordnet.

Das Problem P1 des Aufwandes der Implementierung eines eigenen Systems kann durch das Verwenden eines bereits existierenden Systems umgangen werden.
Dieses muss das Wissen \acs{snik}s nutzen, um Fragen zu beantworten.

Das Ziel unterstützt die Lösung des Problems P2 der mangelhaften anderen Möglichkeiten zur Wissensabfrage oder -aneignung in der Medizininformatik,
und zwar das Ziel des Erreichens eines semantischen Question Answering-Systems, welches Fragen mit hoher Genauigkeit beantworten kann.

\begin{itemize}
	\item Ziel Z1: Benchmark für Semantisches Question Answering für \acs{snik}
	\item Ziel Z2: Semantisches Question Answering-System, welches Fragen zu \acs{snik} mit hoher Qualität beantwortet.
\end{itemize}
\section{Aufgabenstellung}

Den Zielen sind hier jeweils Aufgaben zugeordnet.

\begin{itemize}

	\item Aufgaben zu Ziel Z1:
	\begin{itemize}
		\item Aufgabe A1.1: Sammlung von typischen Benutzerfragen an \acs{snik}
		\item Aufgabe A1.2: Beantwortung dieser Fragen
		\item Aufgabe A1.3: Methodik entwickeln, um ein System anhand dieser Fragen und Antworten zu bewerten (Benchmark)
	\end{itemize}
	\item Aufgaben zu Ziel Z2:
	\begin{itemize}
		\item Aufgabe A2.1: Recherche existierender Semantischer Question Answering Ansätze und Systeme
		\item Aufgabe A2.2: Auswahl eines geeigneten Systems
		\item Aufgabe A2.3: Evaluation dieses Systems am Benchmark
		\item Aufgabe A2.4: Diskussion der Ergebnisse
	\end{itemize}
\end{itemize}

\section{Aufbau der Arbeit}

Einleitend werden in Kapitel 1 die Problemstellung, Ziele und daraus folgende Aufgaben beschrieben.
Danach werden in Kapitel 2 grundlegende Begriffe und für die Arbeit relevante Themengebiete geklärt.
In Kapitel 3 wird der aktuelle Stand der Forschung beschrieben, besonders in Hinsicht auf verschiedene Question Answering-Systeme.
Kapitel 4 beinhaltet die Lösungsansätze für die oben benannten Aufgaben, Kapitel 5 die Ausführung dieser und Kapitel 6 Ergebnisse.
Kapitel 7 ist die kritische Auseinandersetzung mit den Ergebnissen in der Diskussion und ein Ausblick auf mögliche zukünftige Arbeit.
