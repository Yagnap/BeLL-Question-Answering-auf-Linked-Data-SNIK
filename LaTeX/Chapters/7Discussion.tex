%*****************************************
\chapter{Diskussion und Ausblick}\label{ch:discussion}
%*****************************************

\subsection{Benchmark}

In LB: Studenten sollen üben, sich reindenken, damit sich Gelerntes festigt; Eignung für

Future work: mehr Fragen

\subsection{Systeme}

Die Erwartung war, dass das Feld mittlerweile reif genug für Systeme ist, die das einfache und präzise Nutzen von semantischem Question Answering ermöglichen.
Es war aber sehr schwer, Systeme mit ausreichender Dokumentation und bereitgestelltem Programm zu finden, die noch dazu mit eigenen Daten und vielleicht sogar über mehrere Ontologien verwendbar waren.
Funktionieren taten die meisten davon nicht.

Es muss allerdings auch gesagt werden, dass \ac{snik} nicht besonders für semantisches Question Answering geeignet ist.
Dieses geht meist, wie in \cref{sub:qasysteme} gesehen, nach der Methode vor, dass die Prädikate der \ac{sparql}-Abfrage den Prädikaten des Satzes entsprechen sollen.
Da \ac{snik} auf verhältnismäßig wenigen Prädikaten beruht, deren Label noch dazu oft wenig in einem natürlichen Satz verwendet würden, ist es meist unpraktikabel, solche zu verwenden.
An Außnahmen wie \aurl{meta}{isResponsibleForFunction} sieht man, dass diese deutlich besser funktionieren, da sie etwa in der Frage \enquote{What is the CIO responsible for?} vorhanden sind.
Prädikate wie \aurl{meta}{entityTypeComponent} mit dem Label \enquote{entity type component} haben es dort schwerer, da sie auf das Metamodell \ac{snik}s zurückgreifen
und nicht den normalen Sprachgebrauch wiederspiegeln.

Es ist nicht, wie anfangs gewollt, gelungen, ein semantisches Question Answering-System über alle drei Teilontologien zuverlässig verwendbar zu machen,
jedoch ist es gelungen, ein Question Answering-System über eine Teilontologie zumindest theoretisch verwendbar zu machen, die Erfolgsrate ist allerdings \todo{ernüchternd/nur mangelhaft/akzeptabel (ohne "allerdings")}
