%*******************************************************
% Danksagung
%*******************************************************
\pdfbookmark[1]{Danksagung}{acknowledgments}
%\begin{flushright}{\slshape
%    Programming today is a race between software engineers \\
%    striving to build bigger and better idiot-proof programs, \\
%    and the universe trying to build bigger and better idiots. \\
%    So far, the universe is winning. } \\ \medskip
%    \textcolor{darkgray}{---~~Rick Cook~~---}
%\end{flushright}



\bigskip

\begingroup
\let\clearpage\relax
\let\cleardoublepage\relax
\let\cleardoublepage\relax
\chapter*{Danksagung}

Zuallererst möchte ich meinem externen Betreuer, Dr. Konrad Höffner, für die sehr gute und wichtige Betreuung bedanken.
Du warst Ansprechpartner bei allen möglichen Fragen und hast sie immer sehr ausführlich erklärt, das Whiteboard hätte ich mir manchmal abfotografieren sollen.
Es war auch immer schön, sich in den Stunden, zu denen ich im Institut war, gegenseitig abzulenken.
Du hast mir nicht nur Rat für meine BeLL, sondern auch für wissenschaftliche Arbeiten im Allgemeinen und für das Leben gegeben, dafür möchte ich auch danke sagen, auch wenn ich gerade letzteres vermutlich nicht immer beachten werde.
Besonders hilfreich waren auch immer deine Verbesserungsvorschläge und quasi konstantes Korrekturlesen, welche die Arbeit um einiges verbessert haben.

Danke auch an Dr. Dennis Diefenbach, der sich viel Zeit genommen hat, um uns die Möglichkeiten von QAnswer genauer zu erklären und uns bei Fragen bezüglich des Systems geholfen hat.
Die Korrektur der SPARQL-Fragen war sehr wichtig bezüglich der Arbeit, und die Erklärung der Konfiguration hat die uns offen stehenden Möglichkeiten sehr erweitert.
Die Hilfe kam außerdem erstaunlich schnell und flexibel bezüglich der direkten Kommunikation.

Unersetzlich war auch Dr. Franziska Jahn für die fachliche Korrektur der Fragen, und das sehr schnell.
Danke!
Noch habe ich die Medizininformatik nicht studiert, da war die Hilfe durch eine Expertin sehr wichtig.

Für die Hilfe bei der Auswahl eines Systems und den Verweis auf TeBaQA und das Question Answering-Leaderboard möchte ich Prof. Dr. Rico Usbeck danken, welcher uns die Welt der aktuellen Question Answering-Systeme eröffnet hat.
Bedanken möchte ich mich auch bei meinem internen Betreuer, Herr Haase, welcher zu Fragen bezüglich der BeLL bereit stand und da auch immer sehr hilfreich war.
Danke auch an meine Eltern, welche diese Arbeit Korrektur gelesen haben.
Meinen Dank schulde ich auch Prof. Dr. Alfred Winter, da er mich an Dr. Konrad Höffner weitergeleitet hat und meine BeLL am IMISE erst ermöglicht hat.

\endgroup
