\begin{acronym}
% A
\acro{api}[API]{Application Programming Interface}
\acroplural{api}[APIs]{Application Programming Interfaces}
% B
\acro{bert}[BERT]{Bidirectional Encoder Representations from Transformers}
% C
\acro{cdqa}[CDQA]{Closed-Domain Question Answering}
\acro{cnn}[CNN]{Convolutional Neural Network}
\acroplural{cnn}[CNNs]{Convolutional Neural Networks}
% D
\acro{dnn}[DNN]{Deep Neural Network}
\acroplural{dnn}[DNNs]{Deep Neural Networks}
% E
\acro{elmo}[ELMo]{Embeddings from Language Models}
% F
% G
\acro{gru}[GRU]{Gated Research Unit}
% H
\acro{html}[HTML]{Hypertext Markup Language}
\acro{http}[HTTP]{Hypertext Transfer Protocol}
% I
\acro{imise}[IMISE]{Institut für Medizinische Informatik, Statistik und Epidemiologie}
\acro{iri}[IRI]{Internationalized Resorce Identifier}
\acroplural{iri}[IRIs]{Internationalized Resorce Identifiers}
% J
\acro{json}[JSON]{JavaScript Object Notation}
% K
\acro{kbqa}[KBQA]{Knowledgebase Question Answering}
% L
\acro{lstm}[LSTM]{Long short-term memory}
% M
\acro{mlm}[MLM]{masked language model}
% N
\acro{nlp}[NLP]{Natural Language Processing}
\acro{nlu}[NLU]{Natural Language Understanding}
\acro{nn}[NN]{Neural Network}
\acroplural{nn}[NNs]{Neural Networks}
% O
\acro{odqa}[ODQA]{Open-Domain Question Answering}
\acro{owl}[OWL]{Web Ontology Language}
% P
% Q
\acro{qald}[QALD]{Question Answering over Linked Data}
% R
\acro{rdf}[RDF]{Resource Development Framework}
\acro{rest}[REST]{Representational State Transfer}
\acro{rnn}[RNN]{Recurrent Neural Network}
\acroplural{rnn}[RNNs]{Recurrent Neural Networks}
% S
\acro{snik}[SNIK]{Semantisches Netz des Informationsmanagements im Krankenhaus}
\acro{sparql}[SPARQL]{SPARQL Protocol and RDF Query Language}
% T
% U
\acro{uri}[URI]{Uniform Resource Identifier}
\acroplural{uri}[URIs]{Uniform Resource Identifiers}
\acro{url}[URL]{Uniform Resource Locator}
\acroplural{url}[URLs]{Uniform Resource Locators}
% V
% W
\acro{w3c}[W3C]{World Wide Web Consortium}
\acro{www}[WWW]{World Wide Web}
% X
\acro{xml}[XML]{Extensible Markup Language}
% Y
% Z
\acro{zsl}[ZSL]{Zero-Shot Learning}
\end{acronym}
%\acused{URL}% Has its own paragraph in the preliminaries.
\acused{snik}% Explained in Related Work, but appears in Introduction & Preliminaries
