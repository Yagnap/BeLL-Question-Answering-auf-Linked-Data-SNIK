% ****************************************************************************************************
% classicthesis-config.tex
% formerly known as loadpackages.sty, classicthesis-ldpkg.sty, and classicthesis-preamble.sty
% Use it at the beginning of your ClassicThesis.tex, or as a LaTeX Preamble
% in your ClassicThesis.{tex,lyx} with % ****************************************************************************************************
% classicthesis-config.tex
% formerly known as loadpackages.sty, classicthesis-ldpkg.sty, and classicthesis-preamble.sty
% Use it at the beginning of your ClassicThesis.tex, or as a LaTeX Preamble
% in your ClassicThesis.{tex,lyx} with % ****************************************************************************************************
% classicthesis-config.tex
% formerly known as loadpackages.sty, classicthesis-ldpkg.sty, and classicthesis-preamble.sty
% Use it at the beginning of your ClassicThesis.tex, or as a LaTeX Preamble
% in your ClassicThesis.{tex,lyx} with % ****************************************************************************************************
% classicthesis-config.tex
% formerly known as loadpackages.sty, classicthesis-ldpkg.sty, and classicthesis-preamble.sty
% Use it at the beginning of your ClassicThesis.tex, or as a LaTeX Preamble
% in your ClassicThesis.{tex,lyx} with \input{classicthesis-config}
% ****************************************************************************************************
% If you like the classicthesis, then I would appreciate a postcard.
% My address can be found in the file ClassicThesis.pdf. A collection
% of the postcards I received so far is available online at
% http://postcards.miede.de
% ****************************************************************************************************


% ****************************************************************************************************
% 0. Set the encoding of your files. UTF-8 is the only sensible encoding nowadays. If you can't read
% äöüßáéçèê∂åëæƒÏ€ then change the encoding setting in your editor, not the line below. If your editor
% does not support utf8 use another editor!
% ****************************************************************************************************
\PassOptionsToPackage{utf8}{inputenc}
  \usepackage{inputenc}

\PassOptionsToPackage{T1}{fontenc} % T2A for cyrillics
  \usepackage{fontenc}



% ****************************************************************************************************
% 1. Configure classicthesis for your needs here, e.g., remove "drafting" below
% in order to deactivate the time-stamp on the pages
% (see ClassicThesis.pdf for more information):
% ****************************************************************************************************
\PassOptionsToPackage{
  drafting=true,    % print version information on the bottom of the pages
  tocaligned=false, % the left column of the toc will be aligned (no indentation)
  dottedtoc=false,  % page numbers in ToC flushed right
  eulerchapternumbers=true, % use AMS Euler for chapter font (otherwise Palatino)
  linedheaders=false,       % chaper headers will have line above and beneath
  floatperchapter=true,     % numbering per chapter for all floats (i.e., Figure 1.1)
  eulermath=false,  % use awesome Euler fonts for mathematical formulae (only with pdfLaTeX)
  beramono=true,    % toggle a nice monospaced font (w/ bold)
  palatino=true,    % deactivate standard font for loading another one, see the last section at the end of this file for suggestions
  style=classicthesis % classicthesis, arsclassica
}{classicthesis}


% ****************************************************************************************************
% 2. Personal data and user ad-hoc commands (insert your own data here)
% ****************************************************************************************************
\newcommand{\myTitle}{Question Answering auf Linked Data SNIK\xspace}
\newcommand{\mySubtitle}{Besondere Lernleistung Hannes Raphael Brunsch\xspace}
\newcommand{\myName}{Hannes Raphael Brunsch\xspace}
\newcommand{\mybirthday}{20.10.2005}
\newcommand{\mybirthtown}{Leipzig}
\newcommand{\mybirthcountry}{Deutschland}
\newcommand{\myProf}{Herr Haase\xspace}
\newcommand{\mySupervisor}{Dr. Konrad Höffner\xspace}

%\newcommand{\myFaculty}{Institut für Mathematik und Informatik\xspace}
%\newcommand{\mySubject}{Informatik}

%\newcommand{\myFaculty}{Faculty of Mathematics and computer science\xspace}
%\newcommand{\mySubject}{Computer Science}

%\newcommand{\myFaculty}{Medical Faculty\xspace}
%\newcommand{\myDepartment}{Institute for Medical Informatics, Statistics and Epidemiology\xspace}
%\newcommand{\myUni}{Leipzig University\xspace}
%\newcommand{\mySubject}{Medical Informatics}

\newcommand{\myFaculty}{Medizinische Fakultät\xspace}
\newcommand{\myDepartment}{Institut für Medizinische Informatik, Statistik und Epidemiologie\xspace}
\newcommand{\mySchool}{Wilhelm-Ostwald-Schule, Gymnasium der Stadt Leipzig\xspace}
\newcommand{\myUni}{Universität Leipzig\xspace}
\newcommand{\mySubject}{Besondere Lernleistung Referenzfach Informatik\xspace}
%\newcommand{\mySubject}{Informatik M. Sc. mit Studienrichtung Medizinische Informatik\xspace}% alter Master
%\newcommand{\mySubject}{Medizininformatik M. Sc.\xspace}% neuer Master

\newcommand{\myLocation}{Leipzig\xspace}
\newcommand{\myTime}{Abgabedatum\xspace}
\newcommand{\myVersion}{\classicthesis}



% ********************************************************************
% Setup, finetuning, and useful commands
% ********************************************************************
\providecommand{\mLyX}{L\kern-.1667em\lower.25em\hbox{Y}\kelastrn-.125emX\@}
\newcommand{\ie}{i.\,e.}
\newcommand{\Ie}{I.\,e.}
\newcommand{\eg}{e.\,g.}
\newcommand{\Eg}{E.\,g.}
% ****************************************************************************************************


% ****************************************************************************************************
% 3. Loading some handy packages
% ****************************************************************************************************

\usepackage{amsthm}
\newtheorem{definition}{Definition}

% ********************************************************************
% Packages with options that might require adjustments
% ********************************************************************
\PassOptionsToPackage{ngerman}{babel} % change this to your language(s), main language last
% Spanish languages need extra options in order to work with this template
%\PassOptionsToPackage{spanish,es-lcroman}{babel}
\usepackage{babel}

\usepackage[thresholdtype=words]{csquotes}
\PassOptionsToPackage{%
  %backend=biber,bibencoding=utf8, %instead of bibtex
  backend=bibtex8,bibencoding=ascii,%
  language=auto,%
  style=authoryear,dashed=false%
  %style=authoryear-comp, % Author 1999, 2010
  %bibstyle=authoryear,dashed=false, % dashed: substitute rep. author with ---
  sorting=nyt, % name, year, title
  maxbibnames=10, % default: 3, et al.
  %backref=true,%
  natbib=true % natbib compatibility mode (\citep and \citet still work)
}{biblatex}
    \usepackage{biblatex}

\PassOptionsToPackage{fleqn}{amsmath}       % math environments and more by the AMS
  \usepackage{amsmath}

% ********************************************************************
% General useful packages
% ********************************************************************
\usepackage{graphicx} %
\usepackage{scrhack} % fix warnings when using KOMA with listings package
\usepackage{xspace} % to get the spacing after macros right
\PassOptionsToPackage{printonlyused,smaller}{acronym}
  \usepackage{acronym} % nice macros for handling all acronyms in the thesis
  %\renewcommand{\bflabel}[1]{{#1}\hfill} % fix the list of acronyms --> no longer working
  %\renewcommand*{\acsfont}[1]{\textsc{#1}}
  %\renewcommand*{\aclabelfont}[1]{\acsfont{#1}}
  %\def\bflabel#1{{#1\hfill}}
  \def\bflabel#1{{\acsfont{#1}\hfill}}
  \def\aclabelfont#1{\acsfont{#1}}
\PassOptionsToPackage{activate={true,nocompatibility},final,tracking=true,kerning=true,spacing=true,factor=1100,stretch=10,shrink=10,final}{microtype}%final-even in draft mode
\usepackage[]{microtype}
\usepackage{siunitx} % numbers with units
% ****************************************************************************************************
%\usepackage{pgfplots} % External TikZ/PGF support (thanks to Andreas Nautsch)
%\usetikzlibrary{external}
%\tikzexternalize[mode=list and make, prefix=ext-tikz/]
% ****************************************************************************************************

% ****************************************************************************************************
% 4. Setup floats: tables, (sub)figures, and captions
% ****************************************************************************************************
\usepackage{booktabs}
\usepackage{tabularx} % better tables that are smaller than the text width
\usepackage{tabulary} % better tables that are wider than the text width
\setlength{\extrarowheight}{3pt} % increase table row height
\newcommand{\tableheadline}[1]{\multicolumn{1}{l}{\spacedlowsmallcaps{#1}}}
\newcommand{\myfloatalign}{\centering} % to be used with each float for alignment
\usepackage{subfig}
\usepackage{pdflscape}%landscape, e.g. for large tables
% ****************************************************************************************************


% ****************************************************************************************************
% 5. Setup code listings
% ****************************************************************************************************
\usepackage{listings}
%\lstset{emph={trueIndex,root},emphstyle=\color{BlueViolet}}%\underbar} % for special keywords
\lstset{language=[LaTeX]Tex,%C++,
  morekeywords={PassOptionsToPackage,selectlanguage},
  keywordstyle=\color{RoyalBlue},%\bfseries,
  basicstyle=\small\ttfamily,
  %identifierstyle=\color{NavyBlue},
  commentstyle=\color{Green}\ttfamily,
  stringstyle=\rmfamily,
  numbers=none,%left,%
  numberstyle=\scriptsize,%\tiny
  stepnumber=5,
  numbersep=8pt,
  showstringspaces=false,
  breaklines=true,
  %frameround=ftff,
  %frame=single,
  belowcaptionskip=.75\baselineskip
  %frame=L
}
% ****************************************************************************************************




% ****************************************************************************************************
% 6. Last calls before the bar closes
% ****************************************************************************************************
% ********************************************************************
% Her Majesty herself
% ********************************************************************
\usepackage[dottedtoc]{classicthesis}


% ********************************************************************
% Fine-tune hyperreferences (hyperref should be called last)
% ********************************************************************
\hypersetup{%
  %draft, % hyperref's draft mode, for printing see below
  colorlinks=true, linktocpage=true, pdfstartpage=3, pdfstartview=FitV,%
  % uncomment the following line if you want to have black links (e.g., for printing)
  %colorlinks=false, linktocpage=false, pdfstartpage=3, pdfstartview=FitV, pdfborder={0 0 0},%
  breaklinks=true, pageanchor=true,%
  pdfpagemode=UseNone, %
  % pdfpagemode=UseOutlines,%
  plainpages=false, bookmarksnumbered, bookmarksopen=true, bookmarksopenlevel=1,%
  hypertexnames=true, pdfhighlight=/O,%nesting=true,%frenchlinks,%
  urlcolor=CTurl, linkcolor=CTlink, citecolor=CTcitation, %pagecolor=RoyalBlue,%
  %urlcolor=Black, linkcolor=Black, citecolor=Black, %pagecolor=Black,%
  pdftitle={\myTitle},%
  pdfauthor={\textcopyright\ \myName, \myUni, \myFaculty},%
  pdfsubject={},%
  pdfkeywords={},%
  pdfcreator={pdfLaTeX},%
  pdfproducer={LaTeX with hyperref and classicthesis}%
}


% ********************************************************************
% Setup autoreferences (hyperref and babel)
% ********************************************************************
% There are some issues regarding autorefnames
% http://www.tex.ac.uk/cgi-bin/texfaq2html?label=latexwords
% you have to redefine the macros for the
% language you use, e.g., american, ngerman
% (as chosen when loading babel/AtBeginDocument)
% ********************************************************************
\makeatletter
\@ifpackageloaded{babel}%
  {%
    \addto\extrasamerican{%
      \renewcommand*{\figureautorefname}{Figure}%
      \renewcommand*{\tableautorefname}{Table}%
      \renewcommand*{\partautorefname}{Part}%
      \renewcommand*{\chapterautorefname}{Chapter}%
      \renewcommand*{\sectionautorefname}{Section}%
      \renewcommand*{\subsectionautorefname}{Section}%
      \renewcommand*{\subsubsectionautorefname}{Section}%
    }%
    \addto\extrasngerman{%
      \renewcommand*{\paragraphautorefname}{Absatz}%
      \renewcommand*{\subparagraphautorefname}{Unterabsatz}%
      \renewcommand*{\footnoteautorefname}{Fu\"snote}%
      \renewcommand*{\FancyVerbLineautorefname}{Zeile}%
      \renewcommand*{\theoremautorefname}{Theorem}%
      \renewcommand*{\appendixautorefname}{Anhang}%
      \renewcommand*{\equationautorefname}{Gleichung}%
      \renewcommand*{\itemautorefname}{Punkt}%
    }%
      % Fix to getting autorefs for subfigures right (thanks to Belinda Vogt for changing the definition)
      \providecommand{\subfigureautorefname}{\figureautorefname}%
    }{\relax}
\makeatother


% ********************************************************************
% Development Stuff
% ********************************************************************
\listfiles
%\PassOptionsToPackage{l2tabu,orthodox,abort}{nag}
%  \usepackage{nag}
%\PassOptionsToPackage{warning, all}{onlyamsmath}
%  \usepackage{onlyamsmath}


% ****************************************************************************************************
% 7. Further adjustments (experimental)
% ****************************************************************************************************
% ********************************************************************
% Changing the text area
% ********************************************************************
%\areaset[current]{312pt}{761pt} % 686 (factor 2.2) + 33 head + 42 head \the\footskip
%\setlength{\marginparwidth}{7em}%
%\setlength{\marginparsep}{2em}%

% ********************************************************************
% Using different fonts
% ********************************************************************
%\usepackage[oldstylenums]{kpfonts} % oldstyle notextcomp
% \usepackage[osf]{libertine}
%\usepackage[light,condensed,math]{iwona}
%\renewcommand{\sfdefault}{iwona}
%\usepackage{lmodern} % <-- no osf support :-(
%\usepackage{cfr-lm} %
%\usepackage[urw-garamond]{mathdesign} <-- no osf support :-(
%\usepackage[default,osfigures]{opensans} % scale=0.95
%\usepackage[sfdefault]{FiraSans}
% \usepackage[opticals,mathlf]{MinionPro} % onlytext
% ********************************************************************
%\usepackage[largesc,osf]{newpxtext}
%\linespread{1.05} % a bit more for Palatino
% Used to fix these:
% https://bitbucket.org/amiede/classicthesis/issues/139/italics-in-pallatino-capitals-chapter
% https://bitbucket.org/amiede/classicthesis/issues/45/problema-testatine-su-classicthesis-style
% ********************************************************************
% ****************************************************************************************************

% ****** Custom
\usepackage{todonotes}
\usepackage{cleveref}

% ****************************************************************************************************
% If you like the classicthesis, then I would appreciate a postcard.
% My address can be found in the file ClassicThesis.pdf. A collection
% of the postcards I received so far is available online at
% http://postcards.miede.de
% ****************************************************************************************************


% ****************************************************************************************************
% 0. Set the encoding of your files. UTF-8 is the only sensible encoding nowadays. If you can't read
% äöüßáéçèê∂åëæƒÏ€ then change the encoding setting in your editor, not the line below. If your editor
% does not support utf8 use another editor!
% ****************************************************************************************************
\PassOptionsToPackage{utf8}{inputenc}
  \usepackage{inputenc}

\PassOptionsToPackage{T1}{fontenc} % T2A for cyrillics
  \usepackage{fontenc}



% ****************************************************************************************************
% 1. Configure classicthesis for your needs here, e.g., remove "drafting" below
% in order to deactivate the time-stamp on the pages
% (see ClassicThesis.pdf for more information):
% ****************************************************************************************************
\PassOptionsToPackage{
  drafting=true,    % print version information on the bottom of the pages
  tocaligned=false, % the left column of the toc will be aligned (no indentation)
  dottedtoc=false,  % page numbers in ToC flushed right
  eulerchapternumbers=true, % use AMS Euler for chapter font (otherwise Palatino)
  linedheaders=false,       % chaper headers will have line above and beneath
  floatperchapter=true,     % numbering per chapter for all floats (i.e., Figure 1.1)
  eulermath=false,  % use awesome Euler fonts for mathematical formulae (only with pdfLaTeX)
  beramono=true,    % toggle a nice monospaced font (w/ bold)
  palatino=true,    % deactivate standard font for loading another one, see the last section at the end of this file for suggestions
  style=classicthesis % classicthesis, arsclassica
}{classicthesis}


% ****************************************************************************************************
% 2. Personal data and user ad-hoc commands (insert your own data here)
% ****************************************************************************************************
\newcommand{\myTitle}{Question Answering auf Linked Data SNIK\xspace}
\newcommand{\mySubtitle}{Besondere Lernleistung Hannes Raphael Brunsch\xspace}
\newcommand{\myName}{Hannes Raphael Brunsch\xspace}
\newcommand{\mybirthday}{20.10.2005}
\newcommand{\mybirthtown}{Leipzig}
\newcommand{\mybirthcountry}{Deutschland}
\newcommand{\myProf}{Herr Haase\xspace}
\newcommand{\mySupervisor}{Dr. Konrad Höffner\xspace}

%\newcommand{\myFaculty}{Institut für Mathematik und Informatik\xspace}
%\newcommand{\mySubject}{Informatik}

%\newcommand{\myFaculty}{Faculty of Mathematics and computer science\xspace}
%\newcommand{\mySubject}{Computer Science}

%\newcommand{\myFaculty}{Medical Faculty\xspace}
%\newcommand{\myDepartment}{Institute for Medical Informatics, Statistics and Epidemiology\xspace}
%\newcommand{\myUni}{Leipzig University\xspace}
%\newcommand{\mySubject}{Medical Informatics}

\newcommand{\myFaculty}{Medizinische Fakultät\xspace}
\newcommand{\myDepartment}{Institut für Medizinische Informatik, Statistik und Epidemiologie\xspace}
\newcommand{\mySchool}{Wilhelm-Ostwald-Schule, Gymnasium der Stadt Leipzig\xspace}
\newcommand{\myUni}{Universität Leipzig\xspace}
\newcommand{\mySubject}{Besondere Lernleistung Referenzfach Informatik\xspace}
%\newcommand{\mySubject}{Informatik M. Sc. mit Studienrichtung Medizinische Informatik\xspace}% alter Master
%\newcommand{\mySubject}{Medizininformatik M. Sc.\xspace}% neuer Master

\newcommand{\myLocation}{Leipzig\xspace}
\newcommand{\myTime}{Abgabedatum\xspace}
\newcommand{\myVersion}{\classicthesis}



% ********************************************************************
% Setup, finetuning, and useful commands
% ********************************************************************
\providecommand{\mLyX}{L\kern-.1667em\lower.25em\hbox{Y}\kelastrn-.125emX\@}
\newcommand{\ie}{i.\,e.}
\newcommand{\Ie}{I.\,e.}
\newcommand{\eg}{e.\,g.}
\newcommand{\Eg}{E.\,g.}
% ****************************************************************************************************


% ****************************************************************************************************
% 3. Loading some handy packages
% ****************************************************************************************************

\usepackage{amsthm}
\newtheorem{definition}{Definition}

% ********************************************************************
% Packages with options that might require adjustments
% ********************************************************************
\PassOptionsToPackage{ngerman}{babel} % change this to your language(s), main language last
% Spanish languages need extra options in order to work with this template
%\PassOptionsToPackage{spanish,es-lcroman}{babel}
\usepackage{babel}

\usepackage[thresholdtype=words]{csquotes}
\PassOptionsToPackage{%
  %backend=biber,bibencoding=utf8, %instead of bibtex
  backend=bibtex8,bibencoding=ascii,%
  language=auto,%
  style=authoryear,dashed=false%
  %style=authoryear-comp, % Author 1999, 2010
  %bibstyle=authoryear,dashed=false, % dashed: substitute rep. author with ---
  sorting=nyt, % name, year, title
  maxbibnames=10, % default: 3, et al.
  %backref=true,%
  natbib=true % natbib compatibility mode (\citep and \citet still work)
}{biblatex}
    \usepackage{biblatex}

\PassOptionsToPackage{fleqn}{amsmath}       % math environments and more by the AMS
  \usepackage{amsmath}

% ********************************************************************
% General useful packages
% ********************************************************************
\usepackage{graphicx} %
\usepackage{scrhack} % fix warnings when using KOMA with listings package
\usepackage{xspace} % to get the spacing after macros right
\PassOptionsToPackage{printonlyused,smaller}{acronym}
  \usepackage{acronym} % nice macros for handling all acronyms in the thesis
  %\renewcommand{\bflabel}[1]{{#1}\hfill} % fix the list of acronyms --> no longer working
  %\renewcommand*{\acsfont}[1]{\textsc{#1}}
  %\renewcommand*{\aclabelfont}[1]{\acsfont{#1}}
  %\def\bflabel#1{{#1\hfill}}
  \def\bflabel#1{{\acsfont{#1}\hfill}}
  \def\aclabelfont#1{\acsfont{#1}}
\PassOptionsToPackage{activate={true,nocompatibility},final,tracking=true,kerning=true,spacing=true,factor=1100,stretch=10,shrink=10,final}{microtype}%final-even in draft mode
\usepackage[]{microtype}
\usepackage{siunitx} % numbers with units
% ****************************************************************************************************
%\usepackage{pgfplots} % External TikZ/PGF support (thanks to Andreas Nautsch)
%\usetikzlibrary{external}
%\tikzexternalize[mode=list and make, prefix=ext-tikz/]
% ****************************************************************************************************

% ****************************************************************************************************
% 4. Setup floats: tables, (sub)figures, and captions
% ****************************************************************************************************
\usepackage{booktabs}
\usepackage{tabularx} % better tables that are smaller than the text width
\usepackage{tabulary} % better tables that are wider than the text width
\setlength{\extrarowheight}{3pt} % increase table row height
\newcommand{\tableheadline}[1]{\multicolumn{1}{l}{\spacedlowsmallcaps{#1}}}
\newcommand{\myfloatalign}{\centering} % to be used with each float for alignment
\usepackage{subfig}
\usepackage{pdflscape}%landscape, e.g. for large tables
% ****************************************************************************************************


% ****************************************************************************************************
% 5. Setup code listings
% ****************************************************************************************************
\usepackage{listings}
%\lstset{emph={trueIndex,root},emphstyle=\color{BlueViolet}}%\underbar} % for special keywords
\lstset{language=[LaTeX]Tex,%C++,
  morekeywords={PassOptionsToPackage,selectlanguage},
  keywordstyle=\color{RoyalBlue},%\bfseries,
  basicstyle=\small\ttfamily,
  %identifierstyle=\color{NavyBlue},
  commentstyle=\color{Green}\ttfamily,
  stringstyle=\rmfamily,
  numbers=none,%left,%
  numberstyle=\scriptsize,%\tiny
  stepnumber=5,
  numbersep=8pt,
  showstringspaces=false,
  breaklines=true,
  %frameround=ftff,
  %frame=single,
  belowcaptionskip=.75\baselineskip
  %frame=L
}
% ****************************************************************************************************




% ****************************************************************************************************
% 6. Last calls before the bar closes
% ****************************************************************************************************
% ********************************************************************
% Her Majesty herself
% ********************************************************************
\usepackage[dottedtoc]{classicthesis}


% ********************************************************************
% Fine-tune hyperreferences (hyperref should be called last)
% ********************************************************************
\hypersetup{%
  %draft, % hyperref's draft mode, for printing see below
  colorlinks=true, linktocpage=true, pdfstartpage=3, pdfstartview=FitV,%
  % uncomment the following line if you want to have black links (e.g., for printing)
  %colorlinks=false, linktocpage=false, pdfstartpage=3, pdfstartview=FitV, pdfborder={0 0 0},%
  breaklinks=true, pageanchor=true,%
  pdfpagemode=UseNone, %
  % pdfpagemode=UseOutlines,%
  plainpages=false, bookmarksnumbered, bookmarksopen=true, bookmarksopenlevel=1,%
  hypertexnames=true, pdfhighlight=/O,%nesting=true,%frenchlinks,%
  urlcolor=CTurl, linkcolor=CTlink, citecolor=CTcitation, %pagecolor=RoyalBlue,%
  %urlcolor=Black, linkcolor=Black, citecolor=Black, %pagecolor=Black,%
  pdftitle={\myTitle},%
  pdfauthor={\textcopyright\ \myName, \myUni, \myFaculty},%
  pdfsubject={},%
  pdfkeywords={},%
  pdfcreator={pdfLaTeX},%
  pdfproducer={LaTeX with hyperref and classicthesis}%
}


% ********************************************************************
% Setup autoreferences (hyperref and babel)
% ********************************************************************
% There are some issues regarding autorefnames
% http://www.tex.ac.uk/cgi-bin/texfaq2html?label=latexwords
% you have to redefine the macros for the
% language you use, e.g., american, ngerman
% (as chosen when loading babel/AtBeginDocument)
% ********************************************************************
\makeatletter
\@ifpackageloaded{babel}%
  {%
    \addto\extrasamerican{%
      \renewcommand*{\figureautorefname}{Figure}%
      \renewcommand*{\tableautorefname}{Table}%
      \renewcommand*{\partautorefname}{Part}%
      \renewcommand*{\chapterautorefname}{Chapter}%
      \renewcommand*{\sectionautorefname}{Section}%
      \renewcommand*{\subsectionautorefname}{Section}%
      \renewcommand*{\subsubsectionautorefname}{Section}%
    }%
    \addto\extrasngerman{%
      \renewcommand*{\paragraphautorefname}{Absatz}%
      \renewcommand*{\subparagraphautorefname}{Unterabsatz}%
      \renewcommand*{\footnoteautorefname}{Fu\"snote}%
      \renewcommand*{\FancyVerbLineautorefname}{Zeile}%
      \renewcommand*{\theoremautorefname}{Theorem}%
      \renewcommand*{\appendixautorefname}{Anhang}%
      \renewcommand*{\equationautorefname}{Gleichung}%
      \renewcommand*{\itemautorefname}{Punkt}%
    }%
      % Fix to getting autorefs for subfigures right (thanks to Belinda Vogt for changing the definition)
      \providecommand{\subfigureautorefname}{\figureautorefname}%
    }{\relax}
\makeatother


% ********************************************************************
% Development Stuff
% ********************************************************************
\listfiles
%\PassOptionsToPackage{l2tabu,orthodox,abort}{nag}
%  \usepackage{nag}
%\PassOptionsToPackage{warning, all}{onlyamsmath}
%  \usepackage{onlyamsmath}


% ****************************************************************************************************
% 7. Further adjustments (experimental)
% ****************************************************************************************************
% ********************************************************************
% Changing the text area
% ********************************************************************
%\areaset[current]{312pt}{761pt} % 686 (factor 2.2) + 33 head + 42 head \the\footskip
%\setlength{\marginparwidth}{7em}%
%\setlength{\marginparsep}{2em}%

% ********************************************************************
% Using different fonts
% ********************************************************************
%\usepackage[oldstylenums]{kpfonts} % oldstyle notextcomp
% \usepackage[osf]{libertine}
%\usepackage[light,condensed,math]{iwona}
%\renewcommand{\sfdefault}{iwona}
%\usepackage{lmodern} % <-- no osf support :-(
%\usepackage{cfr-lm} %
%\usepackage[urw-garamond]{mathdesign} <-- no osf support :-(
%\usepackage[default,osfigures]{opensans} % scale=0.95
%\usepackage[sfdefault]{FiraSans}
% \usepackage[opticals,mathlf]{MinionPro} % onlytext
% ********************************************************************
%\usepackage[largesc,osf]{newpxtext}
%\linespread{1.05} % a bit more for Palatino
% Used to fix these:
% https://bitbucket.org/amiede/classicthesis/issues/139/italics-in-pallatino-capitals-chapter
% https://bitbucket.org/amiede/classicthesis/issues/45/problema-testatine-su-classicthesis-style
% ********************************************************************
% ****************************************************************************************************

% ****** Custom
\usepackage{todonotes}
\usepackage{cleveref}

% ****************************************************************************************************
% If you like the classicthesis, then I would appreciate a postcard.
% My address can be found in the file ClassicThesis.pdf. A collection
% of the postcards I received so far is available online at
% http://postcards.miede.de
% ****************************************************************************************************


% ****************************************************************************************************
% 0. Set the encoding of your files. UTF-8 is the only sensible encoding nowadays. If you can't read
% äöüßáéçèê∂åëæƒÏ€ then change the encoding setting in your editor, not the line below. If your editor
% does not support utf8 use another editor!
% ****************************************************************************************************
\PassOptionsToPackage{utf8}{inputenc}
  \usepackage{inputenc}

\PassOptionsToPackage{T1}{fontenc} % T2A for cyrillics
  \usepackage{fontenc}



% ****************************************************************************************************
% 1. Configure classicthesis for your needs here, e.g., remove "drafting" below
% in order to deactivate the time-stamp on the pages
% (see ClassicThesis.pdf for more information):
% ****************************************************************************************************
\PassOptionsToPackage{
  drafting=true,    % print version information on the bottom of the pages
  tocaligned=false, % the left column of the toc will be aligned (no indentation)
  dottedtoc=false,  % page numbers in ToC flushed right
  eulerchapternumbers=true, % use AMS Euler for chapter font (otherwise Palatino)
  linedheaders=false,       % chaper headers will have line above and beneath
  floatperchapter=true,     % numbering per chapter for all floats (i.e., Figure 1.1)
  eulermath=false,  % use awesome Euler fonts for mathematical formulae (only with pdfLaTeX)
  beramono=true,    % toggle a nice monospaced font (w/ bold)
  palatino=true,    % deactivate standard font for loading another one, see the last section at the end of this file for suggestions
  style=classicthesis % classicthesis, arsclassica
}{classicthesis}


% ****************************************************************************************************
% 2. Personal data and user ad-hoc commands (insert your own data here)
% ****************************************************************************************************
\newcommand{\myTitle}{Question Answering auf Linked Data SNIK\xspace}
\newcommand{\mySubtitle}{Besondere Lernleistung Hannes Raphael Brunsch\xspace}
\newcommand{\myName}{Hannes Raphael Brunsch\xspace}
\newcommand{\mybirthday}{20.10.2005}
\newcommand{\mybirthtown}{Leipzig}
\newcommand{\mybirthcountry}{Deutschland}
\newcommand{\myProf}{Herr Haase\xspace}
\newcommand{\mySupervisor}{Dr. Konrad Höffner\xspace}

%\newcommand{\myFaculty}{Institut für Mathematik und Informatik\xspace}
%\newcommand{\mySubject}{Informatik}

%\newcommand{\myFaculty}{Faculty of Mathematics and computer science\xspace}
%\newcommand{\mySubject}{Computer Science}

%\newcommand{\myFaculty}{Medical Faculty\xspace}
%\newcommand{\myDepartment}{Institute for Medical Informatics, Statistics and Epidemiology\xspace}
%\newcommand{\myUni}{Leipzig University\xspace}
%\newcommand{\mySubject}{Medical Informatics}

\newcommand{\myFaculty}{Medizinische Fakultät\xspace}
\newcommand{\myDepartment}{Institut für Medizinische Informatik, Statistik und Epidemiologie\xspace}
\newcommand{\mySchool}{Wilhelm-Ostwald-Schule, Gymnasium der Stadt Leipzig\xspace}
\newcommand{\myUni}{Universität Leipzig\xspace}
\newcommand{\mySubject}{Besondere Lernleistung Referenzfach Informatik\xspace}
%\newcommand{\mySubject}{Informatik M. Sc. mit Studienrichtung Medizinische Informatik\xspace}% alter Master
%\newcommand{\mySubject}{Medizininformatik M. Sc.\xspace}% neuer Master

\newcommand{\myLocation}{Leipzig\xspace}
\newcommand{\myTime}{Abgabedatum\xspace}
\newcommand{\myVersion}{\classicthesis}



% ********************************************************************
% Setup, finetuning, and useful commands
% ********************************************************************
\providecommand{\mLyX}{L\kern-.1667em\lower.25em\hbox{Y}\kelastrn-.125emX\@}
\newcommand{\ie}{i.\,e.}
\newcommand{\Ie}{I.\,e.}
\newcommand{\eg}{e.\,g.}
\newcommand{\Eg}{E.\,g.}
% ****************************************************************************************************


% ****************************************************************************************************
% 3. Loading some handy packages
% ****************************************************************************************************

\usepackage{amsthm}
\newtheorem{definition}{Definition}

% ********************************************************************
% Packages with options that might require adjustments
% ********************************************************************
\PassOptionsToPackage{ngerman}{babel} % change this to your language(s), main language last
% Spanish languages need extra options in order to work with this template
%\PassOptionsToPackage{spanish,es-lcroman}{babel}
\usepackage{babel}

\usepackage[thresholdtype=words]{csquotes}
\PassOptionsToPackage{%
  %backend=biber,bibencoding=utf8, %instead of bibtex
  backend=bibtex8,bibencoding=ascii,%
  language=auto,%
  style=authoryear,dashed=false%
  %style=authoryear-comp, % Author 1999, 2010
  %bibstyle=authoryear,dashed=false, % dashed: substitute rep. author with ---
  sorting=nyt, % name, year, title
  maxbibnames=10, % default: 3, et al.
  %backref=true,%
  natbib=true % natbib compatibility mode (\citep and \citet still work)
}{biblatex}
    \usepackage{biblatex}

\PassOptionsToPackage{fleqn}{amsmath}       % math environments and more by the AMS
  \usepackage{amsmath}

% ********************************************************************
% General useful packages
% ********************************************************************
\usepackage{graphicx} %
\usepackage{scrhack} % fix warnings when using KOMA with listings package
\usepackage{xspace} % to get the spacing after macros right
\PassOptionsToPackage{printonlyused,smaller}{acronym}
  \usepackage{acronym} % nice macros for handling all acronyms in the thesis
  %\renewcommand{\bflabel}[1]{{#1}\hfill} % fix the list of acronyms --> no longer working
  %\renewcommand*{\acsfont}[1]{\textsc{#1}}
  %\renewcommand*{\aclabelfont}[1]{\acsfont{#1}}
  %\def\bflabel#1{{#1\hfill}}
  \def\bflabel#1{{\acsfont{#1}\hfill}}
  \def\aclabelfont#1{\acsfont{#1}}
\PassOptionsToPackage{activate={true,nocompatibility},final,tracking=true,kerning=true,spacing=true,factor=1100,stretch=10,shrink=10,final}{microtype}%final-even in draft mode
\usepackage[]{microtype}
\usepackage{siunitx} % numbers with units
% ****************************************************************************************************
%\usepackage{pgfplots} % External TikZ/PGF support (thanks to Andreas Nautsch)
%\usetikzlibrary{external}
%\tikzexternalize[mode=list and make, prefix=ext-tikz/]
% ****************************************************************************************************

% ****************************************************************************************************
% 4. Setup floats: tables, (sub)figures, and captions
% ****************************************************************************************************
\usepackage{booktabs}
\usepackage{tabularx} % better tables that are smaller than the text width
\usepackage{tabulary} % better tables that are wider than the text width
\setlength{\extrarowheight}{3pt} % increase table row height
\newcommand{\tableheadline}[1]{\multicolumn{1}{l}{\spacedlowsmallcaps{#1}}}
\newcommand{\myfloatalign}{\centering} % to be used with each float for alignment
\usepackage{subfig}
\usepackage{pdflscape}%landscape, e.g. for large tables
% ****************************************************************************************************


% ****************************************************************************************************
% 5. Setup code listings
% ****************************************************************************************************
\usepackage{listings}
%\lstset{emph={trueIndex,root},emphstyle=\color{BlueViolet}}%\underbar} % for special keywords
\lstset{language=[LaTeX]Tex,%C++,
  morekeywords={PassOptionsToPackage,selectlanguage},
  keywordstyle=\color{RoyalBlue},%\bfseries,
  basicstyle=\small\ttfamily,
  %identifierstyle=\color{NavyBlue},
  commentstyle=\color{Green}\ttfamily,
  stringstyle=\rmfamily,
  numbers=none,%left,%
  numberstyle=\scriptsize,%\tiny
  stepnumber=5,
  numbersep=8pt,
  showstringspaces=false,
  breaklines=true,
  %frameround=ftff,
  %frame=single,
  belowcaptionskip=.75\baselineskip
  %frame=L
}
% ****************************************************************************************************




% ****************************************************************************************************
% 6. Last calls before the bar closes
% ****************************************************************************************************
% ********************************************************************
% Her Majesty herself
% ********************************************************************
\usepackage[dottedtoc]{classicthesis}


% ********************************************************************
% Fine-tune hyperreferences (hyperref should be called last)
% ********************************************************************
\hypersetup{%
  %draft, % hyperref's draft mode, for printing see below
  colorlinks=true, linktocpage=true, pdfstartpage=3, pdfstartview=FitV,%
  % uncomment the following line if you want to have black links (e.g., for printing)
  %colorlinks=false, linktocpage=false, pdfstartpage=3, pdfstartview=FitV, pdfborder={0 0 0},%
  breaklinks=true, pageanchor=true,%
  pdfpagemode=UseNone, %
  % pdfpagemode=UseOutlines,%
  plainpages=false, bookmarksnumbered, bookmarksopen=true, bookmarksopenlevel=1,%
  hypertexnames=true, pdfhighlight=/O,%nesting=true,%frenchlinks,%
  urlcolor=CTurl, linkcolor=CTlink, citecolor=CTcitation, %pagecolor=RoyalBlue,%
  %urlcolor=Black, linkcolor=Black, citecolor=Black, %pagecolor=Black,%
  pdftitle={\myTitle},%
  pdfauthor={\textcopyright\ \myName, \myUni, \myFaculty},%
  pdfsubject={},%
  pdfkeywords={},%
  pdfcreator={pdfLaTeX},%
  pdfproducer={LaTeX with hyperref and classicthesis}%
}


% ********************************************************************
% Setup autoreferences (hyperref and babel)
% ********************************************************************
% There are some issues regarding autorefnames
% http://www.tex.ac.uk/cgi-bin/texfaq2html?label=latexwords
% you have to redefine the macros for the
% language you use, e.g., american, ngerman
% (as chosen when loading babel/AtBeginDocument)
% ********************************************************************
\makeatletter
\@ifpackageloaded{babel}%
  {%
    \addto\extrasamerican{%
      \renewcommand*{\figureautorefname}{Figure}%
      \renewcommand*{\tableautorefname}{Table}%
      \renewcommand*{\partautorefname}{Part}%
      \renewcommand*{\chapterautorefname}{Chapter}%
      \renewcommand*{\sectionautorefname}{Section}%
      \renewcommand*{\subsectionautorefname}{Section}%
      \renewcommand*{\subsubsectionautorefname}{Section}%
    }%
    \addto\extrasngerman{%
      \renewcommand*{\paragraphautorefname}{Absatz}%
      \renewcommand*{\subparagraphautorefname}{Unterabsatz}%
      \renewcommand*{\footnoteautorefname}{Fu\"snote}%
      \renewcommand*{\FancyVerbLineautorefname}{Zeile}%
      \renewcommand*{\theoremautorefname}{Theorem}%
      \renewcommand*{\appendixautorefname}{Anhang}%
      \renewcommand*{\equationautorefname}{Gleichung}%
      \renewcommand*{\itemautorefname}{Punkt}%
    }%
      % Fix to getting autorefs for subfigures right (thanks to Belinda Vogt for changing the definition)
      \providecommand{\subfigureautorefname}{\figureautorefname}%
    }{\relax}
\makeatother


% ********************************************************************
% Development Stuff
% ********************************************************************
\listfiles
%\PassOptionsToPackage{l2tabu,orthodox,abort}{nag}
%  \usepackage{nag}
%\PassOptionsToPackage{warning, all}{onlyamsmath}
%  \usepackage{onlyamsmath}


% ****************************************************************************************************
% 7. Further adjustments (experimental)
% ****************************************************************************************************
% ********************************************************************
% Changing the text area
% ********************************************************************
%\areaset[current]{312pt}{761pt} % 686 (factor 2.2) + 33 head + 42 head \the\footskip
%\setlength{\marginparwidth}{7em}%
%\setlength{\marginparsep}{2em}%

% ********************************************************************
% Using different fonts
% ********************************************************************
%\usepackage[oldstylenums]{kpfonts} % oldstyle notextcomp
% \usepackage[osf]{libertine}
%\usepackage[light,condensed,math]{iwona}
%\renewcommand{\sfdefault}{iwona}
%\usepackage{lmodern} % <-- no osf support :-(
%\usepackage{cfr-lm} %
%\usepackage[urw-garamond]{mathdesign} <-- no osf support :-(
%\usepackage[default,osfigures]{opensans} % scale=0.95
%\usepackage[sfdefault]{FiraSans}
% \usepackage[opticals,mathlf]{MinionPro} % onlytext
% ********************************************************************
%\usepackage[largesc,osf]{newpxtext}
%\linespread{1.05} % a bit more for Palatino
% Used to fix these:
% https://bitbucket.org/amiede/classicthesis/issues/139/italics-in-pallatino-capitals-chapter
% https://bitbucket.org/amiede/classicthesis/issues/45/problema-testatine-su-classicthesis-style
% ********************************************************************
% ****************************************************************************************************

% ****** Custom
\usepackage{todonotes}
\usepackage{cleveref}

% ****************************************************************************************************
% If you like the classicthesis, then I would appreciate a postcard.
% My address can be found in the file ClassicThesis.pdf. A collection
% of the postcards I received so far is available online at
% http://postcards.miede.de
% ****************************************************************************************************


% ****************************************************************************************************
% 0. Set the encoding of your files. UTF-8 is the only sensible encoding nowadays. If you can't read
% äöüßáéçèê∂åëæƒÏ€ then change the encoding setting in your editor, not the line below. If your editor
% does not support utf8 use another editor!
% ****************************************************************************************************
\PassOptionsToPackage{utf8}{inputenc}
  \usepackage{inputenc}

\PassOptionsToPackage{T1}{fontenc} % T2A for cyrillics
  \usepackage{fontenc}



% ****************************************************************************************************
% 1. Configure classicthesis for your needs here, e.g., remove "drafting" below
% in order to deactivate the time-stamp on the pages
% (see ClassicThesis.pdf for more information):
% ****************************************************************************************************
\PassOptionsToPackage{
  drafting=false,    % print version information on the bottom of the pages
  tocaligned=false, % the left column of the toc will be aligned (no indentation)
  dottedtoc=false,  % page numbers in ToC flushed right
  eulerchapternumbers=true, % use AMS Euler for chapter font (otherwise Palatino)
  linedheaders=false,       % chaper headers will have line above and beneath
  floatperchapter=true,     % numbering per chapter for all floats (i.e., Figure 1.1)
  eulermath=false,  % use awesome Euler fonts for mathematical formulae (only with pdfLaTeX)
  beramono=true,    % toggle a nice monospaced font (w/ bold)
  palatino=true,    % deactivate standard font for loading another one, see the last section at the end of this file for suggestions
  style=classicthesis % classicthesis, arsclassica
}{classicthesis}


% ****************************************************************************************************
% 2. Personal data and user ad-hoc commands (insert your own data here)
% ****************************************************************************************************
\newcommand{\myTitle}{Question Answering auf SNIK\xspace}
\newcommand{\mySubtitle}{Besondere Lernleistung Hannes Raphael Brunsch\xspace}
\newcommand{\myName}{Hannes Raphael Brunsch\xspace}
\newcommand{\mybirthday}{20.10.2005}
\newcommand{\mybirthtown}{Leipzig}
\newcommand{\mybirthcountry}{Deutschland}
\newcommand{\myProf}{Herr Michael Haase\xspace}
\newcommand{\mySupervisor}{Dr. Konrad Höffner\xspace}

%\newcommand{\myFaculty}{Institut für Mathematik und Informatik\xspace}
%\newcommand{\mySubject}{Informatik}

%\newcommand{\myFaculty}{Faculty of Mathematics and computer science\xspace}
%\newcommand{\mySubject}{Computer Science}

%\newcommand{\myFaculty}{Medical Faculty\xspace}
%\newcommand{\myDepartment}{Institute for Medical Informatics, Statistics and Epidemiology\xspace}
%\newcommand{\myUni}{Leipzig University\xspace}
%\newcommand{\mySubject}{Medical Informatics}

\newcommand{\myFaculty}{Medizinische Fakultät\xspace}
\newcommand{\myDepartment}{Institut für Medizinische Informatik, Statistik und Epidemiologie\xspace}
\newcommand{\mySchool}{Wilhelm-Ostwald-Schule, Gymnasium der Stadt Leipzig\xspace}
\newcommand{\myUni}{Universität Leipzig\xspace}
\newcommand{\mySubject}{Besondere Lernleistung Referenzfach Informatik\xspace}
%\newcommand{\mySubject}{Informatik M. Sc. mit Studienrichtung Medizinische Informatik\xspace}% alter Master
%\newcommand{\mySubject}{Medizininformatik M. Sc.\xspace}% neuer Master

\newcommand{\myLocation}{Leipzig\xspace}
\newcommand{\myTime}{20. Januar 2023\xspace}
\newcommand{\myVersion}{\classicthesis}



% ********************************************************************
% Setup, finetuning, and useful commands
% ********************************************************************
\providecommand{\mLyX}{L\kern-.1667em\lower.25em\hbox{Y}\kelastrn-.125emX\@}
\newcommand{\ie}{i.\,e.}
\newcommand{\Ie}{I.\,e.}
\newcommand{\eg}{e.\,g.}
\newcommand{\Eg}{E.\,g.}
% ****************************************************************************************************


% ****************************************************************************************************
% 3. Loading some handy packages
% ****************************************************************************************************

\usepackage{amsthm}
\newtheorem{definition}{Definition}

% ********************************************************************
% Packages with options that might require adjustments
% ********************************************************************
\PassOptionsToPackage{ngerman}{babel} % change this to your language(s), main language last
% Spanish languages need extra options in order to work with this template
%\PassOptionsToPackage{spanish,es-lcroman}{babel}
\usepackage{babel}

\usepackage[thresholdtype=words]{csquotes}
\PassOptionsToPackage{%
  %backend=biber,bibencoding=utf8, %instead of bibtex
  backend=bibtex8,bibencoding=ascii,%
  language=auto,%
  style=authoryear,dashed=false%
  %style=authoryear-comp, % Author 1999, 2010
  %bibstyle=authoryear,dashed=false, % dashed: substitute rep. author with ---
  sorting=nyt, % name, year, title
  maxbibnames=10, % default: 3, et al.
  %backref=true,%
  natbib=true % natbib compatibility mode (\citep and \citet still work)
}{biblatex}
    \usepackage[minbibnames=3, maxcitenames=2, mincitenames=2, sorting=nyt]{biblatex}

\PassOptionsToPackage{fleqn}{amsmath}       % math environments and more by the AMS
  \usepackage{amsmath}

% ********************************************************************
% General useful packages
% ********************************************************************
\usepackage{graphicx} %
\usepackage{scrhack} % fix warnings when using KOMA with listings package
\usepackage{xspace} % to get the spacing after macros right
\PassOptionsToPackage{printonlyused,smaller}{acronym}
  \usepackage{acronym} % nice macros for handling all acronyms in the thesis
  %\renewcommand{\bflabel}[1]{{#1}\hfill} % fix the list of acronyms --> no longer working
  %\renewcommand*{\acsfont}[1]{\textsc{#1}}
  %\renewcommand*{\aclabelfont}[1]{\acsfont{#1}}
  %\def\bflabel#1{{#1\hfill}}
  \def\bflabel#1{{\acsfont{#1}\hfill}}
  \def\aclabelfont#1{\acsfont{#1}}
\PassOptionsToPackage{activate={true,nocompatibility},final,tracking=true,kerning=true,spacing=true,factor=1100,stretch=10,shrink=10,final}{microtype}%final-even in draft mode
\usepackage[]{microtype}
\usepackage{siunitx} % numbers with units
\usepackage{rotating}

% ****************************************************************************************************
%\usepackage{pgfplots} % External TikZ/PGF support (thanks to Andreas Nautsch)
%\usetikzlibrary{external}
%\tikzexternalize[mode=list and make, prefix=ext-tikz/]
% ****************************************************************************************************

% ****************************************************************************************************
% 4. Setup floats: tables, (sub)figures, and captions
% ****************************************************************************************************
\usepackage{booktabs}
\usepackage{tabularx} % better tables that are smaller than the text width
\usepackage{tabulary} % better tables that are wider than the text width
\setlength{\extrarowheight}{3pt} % increase table row height
\newcommand{\tableheadline}[1]{\multicolumn{1}{l}{\spacedlowsmallcaps{#1}}}
\newcommand{\myfloatalign}{\centering} % to be used with each float for alignment
\usepackage{subfig}
\usepackage{pdflscape}%landscape, e.g. for large tables
% ****************************************************************************************************


% ****************************************************************************************************
% 5. Setup code listings
% ****************************************************************************************************
\usepackage{listings}
%\lstset{emph={trueIndex,root},emphstyle=\color{BlueViolet}}%\underbar} % for special keywords
\lstset{language=[LaTeX]Tex,%C++,
  morekeywords={PassOptionsToPackage,selectlanguage},
  keywordstyle=\color{RoyalBlue},%\bfseries,
  basicstyle=\small\ttfamily,
  %identifierstyle=\color{NavyBlue},
  commentstyle=\color{Green}\ttfamily,
  stringstyle=\rmfamily,
  numbers=none,%left,%
  numberstyle=\scriptsize,%\tiny
  stepnumber=5,
  numbersep=8pt,
  showstringspaces=false,
  breaklines=true,
  %frameround=ftff,
  %frame=single,
  belowcaptionskip=.75\baselineskip
  %frame=L
}
% ****************************************************************************************************




% ****************************************************************************************************
% 6. Last calls before the bar closes
% ****************************************************************************************************
% ********************************************************************
% Her Majesty herself
% ********************************************************************
\usepackage[dottedtoc]{classicthesis}


% ********************************************************************
% Fine-tune hyperreferences (hyperref should be called last)
% ********************************************************************
\hypersetup{%
  %draft, % hyperref's draft mode, for printing see below
  colorlinks=true, linktocpage=true, pdfstartpage=3, pdfstartview=FitV,%
  % uncomment the following line if you want to have black links (e.g., for printing)
  %colorlinks=false, linktocpage=false, pdfstartpage=3, pdfstartview=FitV, pdfborder={0 0 0},%
  breaklinks=true, pageanchor=true,%
  pdfpagemode=UseNone, %
  % pdfpagemode=UseOutlines,%
  plainpages=false, bookmarksnumbered, bookmarksopen=true, bookmarksopenlevel=1,%
  hypertexnames=true, pdfhighlight=/O,%nesting=true,%frenchlinks,%
  urlcolor=CTurl, linkcolor=CTlink, citecolor=CTcitation, %pagecolor=RoyalBlue,%
  %urlcolor=Black, linkcolor=Black, citecolor=Black, %pagecolor=Black,%
  pdftitle={\myTitle},%
  pdfauthor={\textcopyright\ \myName, \myUni, \myFaculty},%
  pdfsubject={},%
  pdfkeywords={},%
  pdfcreator={pdfLaTeX},%
  pdfproducer={LaTeX with hyperref and classicthesis}%
}


% ********************************************************************
% Setup autoreferences (hyperref and babel)
% ********************************************************************
% There are some issues regarding autorefnames
% http://www.tex.ac.uk/cgi-bin/texfaq2html?label=latexwords
% you have to redefine the macros for the
% language you use, e.g., american, ngerman
% (as chosen when loading babel/AtBeginDocument)
% ********************************************************************
\makeatletter
\@ifpackageloaded{babel}%
  {%
    \addto\extrasamerican{%
      \renewcommand*{\figureautorefname}{Figure}%
      \renewcommand*{\tableautorefname}{Table}%
      \renewcommand*{\partautorefname}{Part}%
      \renewcommand*{\chapterautorefname}{Chapter}%
      \renewcommand*{\sectionautorefname}{Section}%
      \renewcommand*{\subsectionautorefname}{Section}%
      \renewcommand*{\subsubsectionautorefname}{Section}%
    }%
    \addto\extrasngerman{%
      \renewcommand*{\paragraphautorefname}{Absatz}%
      \renewcommand*{\subparagraphautorefname}{Unterabsatz}%
      \renewcommand*{\footnoteautorefname}{Fu\"snote}%
      \renewcommand*{\FancyVerbLineautorefname}{Zeile}%
      \renewcommand*{\theoremautorefname}{Theorem}%
      \renewcommand*{\appendixautorefname}{Anhang}%
      \renewcommand*{\equationautorefname}{Gleichung}%
      \renewcommand*{\itemautorefname}{Punkt}%
    }%
      % Fix to getting autorefs for subfigures right (thanks to Belinda Vogt for changing the definition)
      \providecommand{\subfigureautorefname}{\figureautorefname}%
    }{\relax}
\makeatother


% ********************************************************************
% Development Stuff
% ********************************************************************
\listfiles
%\PassOptionsToPackage{l2tabu,orthodox,abort}{nag}
%  \usepackage{nag}
%\PassOptionsToPackage{warning, all}{onlyamsmath}
%  \usepackage{onlyamsmath}


% ****************************************************************************************************
% 7. Further adjustments (experimental)
% ****************************************************************************************************
% ********************************************************************
% Changing the text area
% ********************************************************************
%\areaset[current]{312pt}{761pt} % 686 (factor 2.2) + 33 head + 42 head \the\footskip
%\setlength{\marginparwidth}{7em}%
%\setlength{\marginparsep}{2em}%

% ********************************************************************
% Using different fonts
% ********************************************************************
%\usepackage[oldstylenums]{kpfonts} % oldstyle notextcomp
% \usepackage[osf]{libertine}
%\usepackage[light,condensed,math]{iwona}
%\renewcommand{\sfdefault}{iwona}
%\usepackage{lmodern} % <-- no osf support :-(
%\usepackage{cfr-lm} %
%\usepackage[urw-garamond]{mathdesign} <-- no osf support :-(
%\usepackage[default,osfigures]{opensans} % scale=0.95
%\usepackage[sfdefault]{FiraSans}
% \usepackage[opticals,mathlf]{MinionPro} % onlytext
% ********************************************************************
%\usepackage[largesc,osf]{newpxtext}
%\linespread{1.05} % a bit more for Palatino
% Used to fix these:
% https://bitbucket.org/amiede/classicthesis/issues/139/italics-in-pallatino-capitals-chapter
% https://bitbucket.org/amiede/classicthesis/issues/45/problema-testatine-su-classicthesis-style
% ********************************************************************
% ****************************************************************************************************

% ****** Custom
\usepackage{todonotes}
\usepackage{cleveref}
\usepackage[1000]{aurl}
\usepackage{longtable}

% AURL Definitions
\daurl{wiki-commons}{http://commons.wikimedia.org/wiki/}
% et al. statt u.a. bei Bibtex
\DefineBibliographyStrings{german}{andothers={et\addabbrvspace al\adddot}}
% Umlaute in Codeblöcken
\lstset{
  literate={ö}{{\"o}}1
           {ä}{{\"a}}1
           {ü}{{\"u}}1
}

% Vorgaben von Schule
\linespread{1.5} % 1.5-Zeilenabstand


% rotation-Befehl
\newcommand{\rot}{\rotatebox{90}}

% checkmarks

\usepackage{pifont}% http://ctan.org/pkg/pifont
\newcommand{\cmark}{\ding{51}}%
\newcommand{\xmark}{\ding{55}}%

% SPARQL-Syntax highlighting
\lstdefinelanguage{SPARQl}{
  language=SQL,
  morecomment=[l][\color{OliveGreen}]{\#},
  morestring=[b][\color{RedOrange}]\",
  morekeywords={PREFIX, a, CONSTRUCT},
  sensitive=true
}

% line breaking symbols in lstlisting
%\lstset{prebreak=\raisebox{0ex}[0ex][0ex]
%        {\ensuremath{\hookleftarrow}}}
\lstset{postbreak=\raisebox{0ex}[0ex][0ex]
        {\ensuremath{\hookrightarrow\space}}}
\lstset{breaklines=true, breakatwhitespace=true}

% TODO: Überschriften Fett

% Seitenränder nach Schulvorgabe
% für doppelseitigen Druck entfernen
\usepackage[
  paper=a4paper,
  left=40mm,
  right=30mm
]{geometry}

% Konfiguration in thesis.tex nach Titlepage
