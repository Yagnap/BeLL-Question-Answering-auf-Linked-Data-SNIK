% !TeX encoding = UTF-8
% !TeX spellcheck = de_DE

%% Dies gibt Warnungen aus, sollten veraltete LaTeX-Befehle verwendet werden
\RequirePackage[l2tabu, orthodox]{nag}

\documentclass[utf8,biblatex]{lni}
\bibliography{../Dokumentation/Bibliography, ../Dokumentation/snik}

%% Schöne Tabellen mittels \toprule, \midrule, \bottomrule
\usepackage{booktabs}

%% Zu Demonstrationszwecken
\usepackage[math]{blindtext}
\usepackage{mwe}

%% Akronyme
\usepackage{acronym}

%% BibLaTeX-Sonderkonfiguration,
%% falls man schnell eine existierende Bibliographie wiederverwenden will, aber nicht die .bib-Datei händisch anpassen möchte.
%% Bitte \iffalse und \fi entfernen, dann ist diese Konfiguration aktiviert.

\AtEveryBibitem{%
  \ifentrytype{article}{%
  }{%
    \clearfield{doi}%
    \clearfield{issn}%
    \clearfield{url}%
    \clearfield{urldate}%
  }%
  \ifentrytype{inproceedings}{%
  }{%
    \clearfield{doi}%
    \clearfield{issn}%
    \clearfield{url}%
    \clearfield{urldate}%
  }%
}

\begin{document}
%%% Mehrere Autoren werden durch \and voneinander getrennt.
%%% Die Fußnote enthält die Adresse sowie eine E-Mail-Adresse.
%%% Das optionale Argument (sofern angegeben) wird für die Kopfzeile verwendet.
\title[Question Answering auf SNIK]{Question Answering auf einer Ontologie des Informationsmanagements im Krankenhaus}
%%%\subtitle{Untertitel / Subtitle} % falls benötigt
\author[Hannes R. Brunsch]% \and Konrad Höffner]
{Hannes R. Brunsch\footnote{Wilhelm-Ostwald-Schule, Gymnasium der Stadt Leipzig, Willi-Bredel-Straße 15, 04279 Leipzig, Deutschland \email{hrbrunsch@gmail.com}}}
%\and  Konrad Höffner\footnote{Universität Leipzig, Institut für Medizinische Informatik, Statistik und Epidemiologie, Härtelstraße 16--18, 04107 Leipzig, Deutsche \email{konrad.hoeffner@uni-leipzig.de}}}
\startpage{11} % Beginn der Seitenzählung für diesen Beitrag
\editor{Gesellschaft für Informatik}    % Namen der Herausgeber
\booktitle{Studierendenkonferenz Informatik} % Name des Tagungsband; optional Kurztitel
\yearofpublication{2023}
%%%\lnidoi{18.18420/provided-by-editor-02} % Falls bekannt
\maketitle

\begin{abstract}
Die \LaTeX-Klasse \texttt{lni} setzt die Layout-Vorgaben für Beiträge in LNI Konferenzbänden um.
Dieses Dokument beschreibt ihre Verwendung und ist ein Beispiel für die entsprechende Darstellung.
Der Abstract ist ein kurzer Überblick über die Arbeit der zwischen 70 und 150 Wörtern lang sein und das Wichtigste enthalten sollte.
Die Formatierung erfolgt automatisch innerhalb des abstract-Bereichs.
\end{abstract}

\begin{keywords}
LNI Guidelines \and \LaTeX Vorlage
\end{keywords}

\begin{acronym}[nogroupskip]
% fix crazy line spacing
\setlength{\parskip}{0ex}
\setlength{\itemsep}{1.5ex}
% A
\acro{afb}[AFB]{Anforderungsbereich}
\acro{api}[API]{Application Programming Interface}
\acroplural{api}[APIs]{Application Programming Interfaces}
% B
\acro{bert}[BERT]{Bidirectional Encoder Representations from Transformers}
\acro{boa}[BOA]{Bootstrapping linked data}
% C
\acro{cdqa}[CDQA]{Closed-Domain Question Answering}
\acro{cnn}[CNN]{Convolutional Neural Network}
\acroplural{cnn}[CNNs]{Convolutional Neural Networks}
% D
\acro{dag}[DAG]{Directed Acyclical Graph}
\acro{dnn}[DNN]{Deep Neural Network}
\acroplural{dnn}[DNNs]{Deep Neural Networks}
% E
\acro{elmo}[ELMo]{Embeddings from Language Models}
% F
\acro{fsl}[FSL]{Few-Shot Learning}
% G
\acro{gru}[GRU]{Gated Research Unit}
% H
\acro{html}[HTML]{Hypertext Markup Language}
\acro{http}[HTTP]{Hypertext Transfer Protocol}
% I
\acro{imise}[IMISE]{Institut für Medizinische Informatik, Statistik und Epidemiologie}
\acro{iri}[IRI]{Internationalized Resorce Identifier}
\acroplural{iri}[IRIs]{Internationalized Resorce Identifiers}
% J
\acro{json}[JSON]{JavaScript Object Notation}
% K
\acro{kbqa}[KBQA]{Knowledgebase Question Answering}
% L
\acro{lstm}[LSTM]{Long short-term memory}
% M
\acro{mlm}[MLM]{Masked Language Model}
% N
\acro{nlp}[NLP]{Natural Language Processing}
\acro{nlu}[NLU]{Natural Language Understanding}
\acro{nn}[NN]{Neural Network}
\acroplural{nn}[NNs]{Neural Networks}
\acro{nqs}[NQS]{Normalized Query Structure}
% O
\acro{odqa}[ODQA]{Open-Domain Question Answering}
\acro{owl}[OWL]{Web Ontology Language}
% P
\acro{pos}[POS]{Part-Of-Speech}
% Q
\acro{qald}[QALD]{Question Answering over Linked Data}
% R
\acro{rdf}[RDF]{Resource Development Framework}
\acro{rest}[REST]{Representational State Transfer}
\acro{rnn}[RNN]{Recurrent Neural Network}
\acroplural{rnn}[RNNs]{Recurrent Neural Networks}
% S
\acro{snik}[SNIK]{Semantisches Netz des Informationsmanagements im Krankenhaus}
\acro{sparql}[SPARQL]{SPARQL Protocol and RDF Query Language}
% T
\acro{turtle}[Turtle]{Terse RDF Triple Language}
% U
\acro{uri}[URI]{Uniform Resource Identifier}
\acroplural{uri}[URIs]{Uniform Resource Identifiers}
\acro{url}[URL]{Uniform Resource Locator}
\acroplural{url}[URLs]{Uniform Resource Locators}
% V
% W
\acro{w3c}[W3C]{World Wide Web Consortium}
\acro{www}[WWW]{World Wide Web}
% X
\acro{xml}[XML]{Extensible Markup Language}
% Y
% Z
\acro{zsl}[ZSL]{Zero-Shot Learning}
\end{acronym}
%\acused{URL}% Has its own paragraph in the preliminaries.
%\acused{snik}% Explained in Related Work, but appears in Introduction & Preliminaries


\section{Einleitung}

\ac{snik} ist ein Projekt zum Informationsmanagement im Gesundheitswesen.
Es fasst Wissen aus drei Lehrbüchern, welches es in semantischer Weise modelliert und publiziert.
In der Ontologie (siehe \cref{sub:ontology}) sind Zusammenhänge zwischen verschiedenen Informationen vorhanden,
welche auf der öffentlich erreichbaren Visualisierung in Form eines Graphen dargestellt sind.

\section{Ergebnisse}

* hier die Plots mit den Kurven möglichst auf einer Seite darstellen
* herausfinden wie das offiziell heißt, Lernkurve?

\section{Diskussion und Ausblick}

Ein Modell ist immer für einen bestimmten Zweck erstellt und bildet nur einen Teil der Wirklichkeit ab.
SNIK fokussiert sich auf das Beantworten der Fragestellung "Wer (Rolle) macht was (Aufgabe) womit (Objekttyp)?".
Solche Fragen beantwortet das trainierte System auch sehr gut, die größte Herausforderung ist allerdings der "Mismatch" im verwendeten Vokabular eines menschlichen Nutzers zu den Properties der Ontologie.
Während in einer Wissensbasis die Abbildung von Verben zu Properties einfacher ist, weicht besonders bei der Beschreibung von Subklassen und Teil-Ganzes-Beziehungen das Vokabular menschlicher Nutzer oft von den Labels in der Ontologie ab.
Insgesamt erreicht das System bei dieser Art von Fragen  einen F-Score von 0.9... auf ..., siehe ...
Fragen, welche nicht diesem Schema folgen, wie die Leseverständnisfragen aus \cite{bb} (fußnote zu link im zenodo archiv), können selbst nach dem Training nur selten richtig beantwortet werden (F-Score von ... auf ..., siehe ...).
Unserer Einschätzung nach ist dies eine grundlegende Limitierung der gewählten Modellierung, wir erwarten daher auch bei zukünftigen Systemen keine überwiegend richtige Beantwortung der Verständnisfragen.
Wir planen zur Beantwortung allgemeiner Fragen, die über "Wer macht was womit" hinausgehen, das Training von Sprachmodellen direkt auf den Lehrbüchern.
Ein hybrides System aus KBQA und Sprachmodell hat das Potenzial, die Stärken beider Ansätze zu vereinen.

%% \bibliography{lni-paper-example-de.tex} ist hier nicht erlaubt: biblatex erwartet dies bei der Preambel
%% Starten Sie "biber paper", um eine Biliographie zu erzeugen.
\printbibliography

\end{document}
