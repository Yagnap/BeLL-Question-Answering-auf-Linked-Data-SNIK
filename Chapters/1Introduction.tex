 %************************************************
\chapter{Einleitung}\label{ch:introduction}
%************************************************

\section{Gegenstand}
\begin{flushright}{\slshape
\enquote{Die Medizinische Informatik ist die Wissenschaft der systematischen Erschließung, Verwaltung, Aufbewahrung, Verarbeitung und Bereitstellung von Daten, Informationen und Wissen in der Medizin und im Gesundheitswesen.
Sie ist von dem Streben geleitet, damit zur Gestaltung der bestmöglichen Gesundheitsversorgung beizutragen.\footnote{\url{https://www.gmds.de/de/aktivitaeten/medizinische-informatik/}, abgerufen am 11. Januar 2022}
}}
\end{flushright}
\ac{snik} ist ein Projekt zum Informationsmanagement im Gesundheitswesen.
Es fasst Wissen aus drei Lehrbüchern, welches es in semantischer Weise modelliert und publiziert.
In der Wissensbasis\todo{Umformulieren!} sind Zusammenhänge und Verbindungen zwischen verschiedenen Informationen vorhanden, welche auf der öffentlich erreichbaren Visualisierung in Form eines Graphen, der diese aufzeigt, dargestellt sind.

Diese Daten liegen strukturiert, aber für Menschen nicht leicht lesbar vor. Daher braucht es andere Methoden, durch sie zu navigieren.
Dazu gibt es verschiedene Lösungen,
\begin{enumerate}
	\item das Verwenden eines \acs{rdf}-Browsers, der \acs{rdf}-Informationen einzeln darstellt,
	\item den \ac{snik}-Graphen oder
	\item \acs{sparql}, eine Sprache Abfragesprache für \acs{rdf}.
\end{enumerate}

\section{Problemstellung}

Momentan müssen Studierende der Medizininformatik, die nach Wissen suchen wollen, eine der drei oben genannten Optionen verwenden.
Alle dieser Möglichkeiten haben jedoch große Nachteile.
Der RDF-Browser gibt nur ein sehr beschränktes Ergebnis aus, und serialisiertes RDF selbst zu lesen ist schwer und unpraktisch.
Im Fall von SPARQL nimmt es den Medizininformatik-Studierenden wertvolle Zeit, da sie sich dort erst in die Syntax des Programms und das Vokabular des Fachbereichs einarbeiten müssen.

Ein Ansatz für die Lösung dieses Problems ist Question Answering.
Dabei wird eine Frage als ganzer Satz eingegeben, dessen Bedeutung das System versteht und auf Grundlage dessen eine Antwort ausgibt.\todo{Quelle finden.}
Die Implementierung eines Question Answering-Systems mit adäquater Qualität der Antworten benötigt allerdings mehr Zeit, als es im Rahmen einer BeLL möglich ist.

\begin{itemize}
	\item Problem P1: Intuitivität und Expressivität
	\item Problem P2: Für die Implementierung von Question Answering benötigter Aufwand
\end{itemize}


\section{Motivation}

\todo{Motivation}

\section{Zielsetzung}\label{sec:zielsetzung}
	\begin{itemize}
		\item Ziel Z1: Benchmark für Semantisches Question Answering für \ac{snik}
		\item Ziel Z2: Semantisches Question Answering System, welches Fragen zu \ac{snik} mit hoher Qualität beantwortet.
	\end{itemize}
\section{Aufgabenstellung}

\begin{itemize}
	\item Aufgaben zu Ziel Z1:
	\begin{itemize}
		\item Aufgabe A1.1: Sammlung von typischen Benutzerfragen an \ac{snik}
		\item Aufgabe A1.2: Beantwortung dieser Fragen
		\item Aufgabe A1.3: Methodik entwickeln, um ein System anhand dieser Fragen und Antworten zu bewerten (Benchmark)
	\end{itemize}
	\item Aufgaben zu Ziel Z2:
	\begin{itemize}
		\item Aufgabe A2.1: Recherche existierender Semantischer Question Answering Ansätze und Systeme
		\item Aufgabe A2.2: Auswahl von mindestens zwei verschiedenen ausführbar Systemen
		\item Aufgabe A2.3: Evaluation dieser Systeme am Benchmark
		\item Aufgabe A2.4: Diskussion der Ergebnisse und Vorschlag eines Kandidaten
	\end{itemize}
\end{itemize}

\section{Aufbau der Arbeit}
