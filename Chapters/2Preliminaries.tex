%*****************************************
\chapter{Grundlagen}\label{ch:preliminaries}
%*****************************************

\section{Medizinische Informatik}

\subsection{Informationssysteme im Gesundheitswesen}

\begin{definition}[System]
\enquote{Ein System ist eine Menge an Elementen und deren Beziehungen.} \citep[S.~30]{bb}
\end{definition}
\begin{definition}[Informationssystem]
\enquote{Ein Informationssystem ist ein System das Daten, Informationen und Wissen speichert und verarbeitet.} \citep[S.~30]{bb}
\end{definition}

\begin{definition}[Informationssystem im Gesundheitswesen]
\enquote{Ein Krankenhausinformationssystem ist das sozioökonomische Subsystem eines Krankenhauses, welches die gesamte Informationsverarbeitung sowie die zugehörigen menschlichen Akteure in deren respektiven Informationsverarbeitungsrollen umfasst.} \citep[S.~37]{bb}
\end{definition}

\subsection{Daten, Informationen, Wissen}
Daten sind die physische Repräsentation von Informationen oder Wissen, also zum Beispiel die Zeichenkette \enquote{\ac{snik}}. Daten können neu interpretiert werden.
Informationen hingegen sind Fakten, die gegebenenfalls aus Daten hervorgehen.
Wissen ist eine generelle Information über ein Konzept. \citep[S.~29]{bb}

\section{Semantic Web}

\subsection{URIs, IRIs und URLs}
\ac{uri} werden zur eindeutigen Zuordnung von Zeichenketten zu beliebigen Ressourcen verwendet \citep{uri}.
\ac{url} ist eine URL mit konkreten Zugriffsmöglichkeiten auf die Ressource \citep{url}.
Ein \ac{iri} ist ein URI mit einer größeren Anzahl an möglichen Zeichen, da es auch nichtlateinische Schriftzeichen erlaubt.

\subsection{World Wide Web}
Das \ac{www} wurde 1991 von Tim Berners-Lee entwickelt.
Es ermöglicht freien Zugriff auf und Austausch von Daten über das \ac{http}.
Die Standards zu dieser Kommunikation setzt das \ac{w3c}, dessen Direktor Berners-Lee ist.
Das Format der Dokumente wird durch die \ac{html} beschrieben, welche ebenfalls vom \ac{w3c} kontrolliert wird \citep{www}.

\subsection{Semantic Web}
Im Gegensatz zu \ac{www}, einem Web der Dokumente, ist das Semantic Web ein Web der Daten.
Es ist außerdem nicht nur ein verwendbares Artefakt wie das \ac{www}, sondern auch ein Forschungsfeld \citep{semanticwebreview}, das sehr stark von anderen Forschungsfeldern, wie etwa Deep Learning und künstlicher Intelligenz, abhängig ist. Teile des Forschungsfeldes sind Ontologien, Linked Data und sogenannte Knowledge Graphs, auf die hier aber nicht weiter eingegangen werden wird.
Das Semantic Web als nutzbares Objekt besteht aus strukturierten Daten, sodass man, im Gegensatz zum \ac{www}, Informationen direkt abfragen kann und nicht erst die HTML-Dokumente auswerten muss.

\subsection{RDF}
Das \ac{rdf} ist das grundlegende zur Beschreibung des Semantic Webs genutzte Format \citep[S.~35]{semanticwebgrundlagen}.
Mithilfe dieser formalen Sprache, die erstmals 1999 durch das \ac{w3c} spezifiziert wurde, sollten zuerst Metadaten modelliert werden.
Aufgrund dem Fokus auf der Weiterverarbeitbarkeit der Daten ist es für seine Nutzer gut zur Modellierung des Semantic Webs geeignet.
RDF wird in Tripeln aus Subjekt, Objekt und Prädikat modelliert.
Es verhält sich hierbei wie in einem normalen deutschen Satz, das Prädikat beschreibt also die Beziehung zwischen dem Subjekt und Objekt.
Alle drei werden normalerweise als \acp{uri} angegeben \citep{linkeddatadesignissues}.
Beispiel für solche Tripel aus \acp{uri} sind in \ref{tab:rdftripleexample} zu sehen:
Wilhelm Ostwald (Subjekt) hat den Preis (Prädikat) Nobelpreis der Chemie (Objekt).
Im zweiten Beispiel ist Leipzig das Subjekt und eine Abbildung des Weihnachtsmarktes das Objekt, was noch einmal verdeutlicht, dass \acp{uri} Ressourcen darstellen.
Das dritte Beispiel zeigt, dass für manche Informationen \acp{uri} nur teilweise angeben, was diese Information ist, hier eine nicht negative Ganzzahl.
Die Informationen wurden von DBPedia heruntergeladen\footnote{\url{https://dbpedia.org/page/Leipzig} und \url{https://dbpedia.org/page/Wilhelm_Ostwald}, abgerufen am 14. Januar 2022}.
\begin{table}[h]
  \tiny
  \begin{tabulary}{\textwidth}{rcl}
    \toprule
    Subjekt & Prädikat & Objekt \\
    \midrule
    <http://dbpedia.org/resource/Wilhelm\_Ostwald> & <http://dbpedia.org/ontology/award> & <http://dbpedia.org/resource/Nobel\_Prize\_in\_Chemistry> \\
    <http://dbpedia.org/resource/Leipzig> & <http://xmlns.com/foaf/0.1/depiction> & <http://commons.wikimedia.org/wiki/Special:FilePath/Leipziger\_Weihnachtsmarkt\_Eingang.jpg> \\
    <http://dbpedia.org/resource/Leipzig> & <http://dbpedia.org/ontology/populationTotal> & "605407"^^<http://www.w3.org/2001/XMLSchema\#nonNegativeInteger> \\
    \bottomrule
  \end{tabulary}
  \caption{Beispiele für RDF-Tripel}
  \label{tab:rdftripleexample}
\end{table}
\todo{Formatierungsfehler Zeile 4 Objekt; Tabelle zentrieren}

\subsection{Ontologien}
\begin{definition}[Ontologie]
\enquote{Im Kontext der Informatik ist eine Ontologie eine Menge von repräsentativen Primitiven, welche einen Bereich des Wissens oder eines Diskurses modellieren.
Die repräsentativen Primitiven sind typischerweise Klassen (oder Mengen), Attribute (oder Eigenschaften), und Zusammenhänge (oder Beziehungen zwischen Klassenelementen).
Die Definitionen der repräsentativen Primitiven enthalten Informationen über ihre Bedeutung und Beschränkungen ihrer logisch konsistenten Anwendung.
Im Kontext von Datenbanksystemen kann eine Ontologie als ein Level der Abstraktion von Datenmodellen, analog zu hierarchischen und relationalen Modellen, gesehen werden, aber dazu gedacht, Wissen über Individuen, deren Attribute, und deren Beziehungen zu anderen Individuen zu modellieren.
Ontologien sind typischerweise in Sprachen, die Abstraktionen weg von Datenstrukturen und Implementationsstrategien ermöglichen spezifiziert;
In Praxis sind die Sprachen von Ontologien in der expressiven Kraft näher an Logik erster Ordnung als alle anderen zur Modellierung von Datenbanken genutzten Sprachen.
Deshalb werden Ontologien als auf dem \enquote{semantischen} Level stehend angesehen, wohingegen andere Datenbankschemen auf dem \enquote{logischen} oder \enquote{physischen} Level sind.
Durch ihre Unabhängigkeit von Datenmodellen niedrigeren Levels werden Ontologien zum Integrieren von heterogenen Datenbanken genutzt, was die Benutzbarkeit über verschiedene Systeme erlaubt, und die Spezifizierung von unabhängigen, wissensbasierten  Dienstleistungen ermöglicht.
Im Technologie-Stack der Semantic Web Standards sind Ontologien ihre eigene Schicht.
Es gibt nun standardisierte Sprachen und eine Vielzahl an kommerziellen und Open-Source Werkzeugen, um Ontologien zu erstellen und mit ihnen zu arbeiten.
}
\end{definition}
\textcolor{darkgray}{Aus dem Englischen von \citet{ontologygruber}}
\subsection{Taxonomie}

Nur Beziehungen Subklassen->Klassen

\subsection{Individuen und Klassen}
Die Beziehung zwischen Individuum und Klasse in der Mathematik ist eine Elementbeziehung
Beziehungen zwischen Klassen ==> Teilmengenbeziehung (nur für Beziehung zw. Unterklassen und Oberklassen)
Beziehungen zwischen Individuen untereinander ==> Relationen
\subsubsection{Relationen}
\todo{Properties in RDF}
Object-Properties (=> Klassen als Wertebereich) und Data-Properties (=> Literale)
In RDF alle Relationen zweistellig (Paare)
\subsubsection{Wissensbasen}
Es wird teilweise auch zwischen A- und TBoxen unterschieden.
Eine ABox enthält Wissen über Individuen bzw. Instanzen, wohingegen eine TBox Wissen über Klassen bzw. generelle Schemen enthält \citep[S.~167]{semanticwebgrundlagen}.
\subsubsection{Unterschied zwischen relationalen Datenbanken und Wissenbasen}



\subsection{Linked (Open) Data}

Linked Data beschreibt öffentlich verfügbare Informationen, die mittels \acp{uri} erreichbar und in für Maschinen lesbarer Form vorhanden sind \citep{linkeddata}.
Es sollen verschiedene open-source lizensierte Datenquellen in \ac{rdf} umgeformt und verbunden werden, was allerdings Probleme im Sinne von fehlender Konsistenz der Daten führen kann \citep{semanticwebreview}.
Berners-Lee verfasste vier Regeln für das veröffentlichen von Daten im Internet, um ein umfassendes Linked Data-System zu ermöglichen \citep{linkeddatadesignissues}.
Diese sagen vor allem, dass klar mit \ac{http}-\acp{uri} benannte Ressourcen standardisiert durch \ac{rdf} und \ac{sparql} weitere nützliche Informationen und Links zu anderen \acp{uri} enthalten sollen.

\subsection{SPARQL}

\ac{sparql} ist eine Abfragesprache für \ac{rdf}, das heißt mit diesem \ac{w3c}-Standard kann man in einer in \ac{rdf} verfassten Ontologie Wissen abfragen.
\todo{für SPARQL braucht man keine Ontologie! RDF und Ontologien sind zwar verbunden aber nicht das gleiche.}
\section{Semantisches Question Answering}

\begin{definition}[Question Answering]
Question Answering (Fragebeantwortung) behandelt die Beantwortung von Benutzerfragen \citep{qadefinition}.
Ein Question Answering-System muss eine Frage analysieren, eine oder mehrere Antworten bereitstellen und dem Nutzer diese präsentieren.
\end{definition}

Es existieren sowohl \emph{closed-domain} als auch \emph{open-domain} Question Answering.
Question Answering-Programme mit offener Domäne sind schwieriger zu entwickeln,
da sie Fragen von allen möglichen Domänen verstehen müssen.
Ein Beispiel hierfür wäre die Frage, wo eine Bank zu finden sei.
Die fragenstellende Person könnte entweder eine Filiale einer Bankgesellschaft oder eine Parkbank meinen.
Das Programm, an das die Frage gestellt wird, muss nun mit wenig Kontext herausfinden,
welche Deutung impliziert wird, was zum Beispiel mit mehr Daten,
wie etwa dem vorherigen Suchverlauf (Hat sich die Person vorher nach Parks/Geldfilialen erkundigt?)
oder persönlichen Daten, wie dem Alter, möglich wäre.

Diese Arbeit beschränkt sich jedoch auf Question Answering mit geschlossener Domäne, \ac{snik} (siehe \ref{ch:relatedWork}).
Solche Question Answering-Programme beschränken sich auf einen Fachbereich,
was etwa den Kontext vom vorherigen Beispiel bereitstellen würde.

\begin{definition}[Semantisches Question Answering]
Semantisches Question Answering ist die Beantwortung von Fragen, die in natürlicher Sprache gestellt wurden.
Ein Programm für Semantisches Question Answering erkennt verschiedene semantische Strukturen in der Frage,
also zum Beispiel was für ein Typ die Antwort auf die Frage sein soll, wie etwa eine Zeit oder ein Ort \citep{sqadefinition}.
Dafür wird \acs{nlp} verwendet.
\end{definition}

\subsection{Neuronales Netz}



\subsection{Deep Learning}



\subsection{Natural Language Processing}

\ac{nlp}

\subsection{NLP-Pipeline}



\subsection{Regelbasiertes System}
