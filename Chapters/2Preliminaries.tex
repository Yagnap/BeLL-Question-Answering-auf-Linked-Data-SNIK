%*****************************************
\chapter{Grundlagen}\label{ch:preliminaries}
%*****************************************

\section{Medizinische Informatik}

\subsection{Informationssysteme im Gesundheitswesen}

\begin{definition}[System]
Ein System ist eine Menge an Elementen und deren Beziehungen. (Winter et al., 2010: 30)\todo{mit LaTeX-Kommandos zitieren nicht als Text. z.B. \citep[S.~30]{bb}}
\end{definition}

\begin{definition}[Informationssystem]
Ein Informationssystem ist ein System das Daten, Informationen und Wissen speichert und verarbeitet. (Winter et al., 2010: 30)\todo{mit LaTeX-Kommandos zitieren nicht als Text.}
\end{definition}

\begin{definition}[Informationssystem im Gesundheitswesen]
\enquote{Ein Krankenhausinformationssystem ist das sozioökonomische Subsystem eines Krankenhauses, welches die gesamte Informationsverarbeitung sowie die zugehörigen menschlichen Akteure in deren respektiven Informationsverarbeitungsrollen umfasst.} (Winter et al., 2010: 37)\todo{mit LaTeX-Kommandos zitieren nicht als Text.}
\end{definition}

\subsection{Daten, Informationen, Wissen}
Daten sind die physische Repräsentation von Informationen oder Wissen, also zum Beispiel die Zeichenkette \enquote{SNIK}. Daten können neu interpretiert werden.
Informationen hingegen sind Fakten, die gegebenenfalls aus Daten hervorgehen.
Wissen ist eine generelle Information über ein Konzept. (Winter et al., 2010: 29)

\section{Semantic Web}

\subsection{URIs, IRIs und URLs}
Uniform Resource Identifier (URI) werden zur eindeutigen Zuordnung von Zeichenketten zu beliebigen Ressourcen verwendet.
\todo{Quellen angeben für alles, z.B. URI IRI, URL, WWW, RDF,.... URI ist z.B. hier definiert: https://www.ietf.org/rfc/rfc2396.txt.
Aber bitte nicht einfach als Fußnote sondern richtig einen bibtex Eintrag erstellen. Typ müsste "techreport" sein.}
Ein Uniform Resource Locator (URL) ist eine URL mit konkreten Zugriffsmöglichkeiten auf die Ressource.
Ein Internationalized Resorce Identifier (IRI) ist ein URI mit einer größeren Anzahl an möglichen Zeichen, da es auch nichtlateinische Schriftzeichen erlaubt.

\subsection{World Wide Web}
Das World Wide Web (WWW) wurde 1991 von Tim Berners-Lee entwickelt.
Es ermöglicht freien Zugriff auf und Austausch von Daten über das Hypertext Transfer Protocol (HTTP).
Die Standards zu dieser Kommunikation setzt das World Wide Web Consortium (W3C), dessen Direktor Berners-Lee ist.\todo{Diese Standards auch alle referenzieren.}
Das Format der Dokumente wird durch die Hypertext Markup Language (HTML) beschrieben, welche ebenfalls vom W3C kontrolliert wird.

\subsection{RDF}
Das Resource Development Framework (RDF) ist das grundlegende zur Beschreibung des Semantic Webs genutzte Format. (Winter et al., 35)
Mithilfe dieser formalen Sprache, die erstmals 1999 durch W3C spezifiziert wurde, sollten zuerst Metadaten modelliert werden.
Aufgrund dem Fokus auf der Weiterverarbeitbarkeit der Daten ist es für seine Nutzer gut zur Modellierung des Semantic Webs geeignet.
RDF wird in Tripeln aus Subjekt, Objekt und Prädikat modelliert. Es verhält sich hierbei wie in einem normalen deutschen Satz.

\subsection{Ontologien}
\begin{definition}[Ontologie]
Im Kontext der Informatik ist eine Ontologie eine Menge von repräsentativen Primitiven, welche einen Bereich des Wissens oder eines Diskurses modellieren.
Die repräsentativen Primitiven sind typischerweise Klassen (oder Mengen), Attribute (oder Eigenschaften), und Zusammenhänge (oder Beziehungen zwischen Klassenelementen).
Die Definitionen der repräsentativen Primitiven enthalten Informationen über ihre Bedeutung und Beschränkungen ihrer logisch konsistenten Anwendung.
Im Kontext von Datenbanksystemen kann eine Ontologie als ein Level der Abstraktion von Datenmodellen, analog zu hierarchischen und relationalen Modellen, gesehen werden, aber dazu gedacht, Wissen über Individuen, deren Attribute, und deren Beziehungen zu anderen Individuen zu modellieren.
Ontologien sind typischerweise in Sprachen, die Abstraktionen weg von Datenstrukturen und Implementationsstrategien ermöglichen spezifiziert;
In Praxis sind die Sprachen von Ontologien in der expressiven Kraft näher an Logik erster Ordnung als alle anderen zur Modellierung von Datenbanken genutzten Sprachen.
Deshalb werden Ontologien als auf dem \enquote{semantischen} Level stehend angesehen, wohingegen andere Datenbankschemen auf dem \enquote{logischen} oder \enquote{physischen} Level sind.
Durch ihre Unabhängigkeit von Datenmodellen niedrigeren Levels werden Ontologien zum Integrieren von heterogenen Datenbanken genutzt, was die Benutzbarkeit über verschiedene Systeme erlaubt, und die Spezifizierung von unabhängigen, wissensbasierten  Dienstleistungen ermöglicht.
Im Technologie-Stack der Semantic Web Standards sind Ontologien ihre eigene Schicht.
Es gibt nun standardisierte Sprachen und eine Vielzahl an kommerziellen und Open-Source Werkzeugen, um Ontologien zu erstellen und mit ihnen zu arbeiten.
\end{definition} (Gruber, 2008)

\subsection{Linked (Open) Data}


\subsection{SPARQL}
SPARQL Protocol and RDF Query Language (SPARQL) ist eine Abfragesprache für RDF, das heißt mit diesem W3C-Standard kann man in einer in RDF verfassten Ontologie Wissen abfragen.\todo{standard referenzieren}

\section{Semantisches Question Answering}

\begin{definition}{Question Answering}

\end{definition}

\begin{definition}{Semantisches Question Answering}

\end{definition}
