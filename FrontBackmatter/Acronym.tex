\begin{acronym}
% A
% B
\acro{bert}[BERT]{Bidirectional Encoder Representations from Transformers}
% C
% D
\acro{dnn}[DNN]{Tiefes Neuronales Netz}
\acroplural{dnn}[DNNs]{Tiefe Neuronale Netze}
% E
\acro{elmo}[ELMo]{Embeddings from Language Models}
% F
% G
% H
\acro{html}[HTML]{Hypertext Markup Language}
\acro{http}[HTTP]{Hypertext Transfer Protocol}
% I
\acro{iri}[IRI]{Internationalized Resorce Identifier}
\acroplural{iri}[IRIs]{Internationalized Resorce Identifiers}
% J
% K
% L
% M
% N
\acro{nlp}[NLP]{Natural Language Processing}
\acro{nn}[NN]{Neuronales Netz}
\acroplural{nn}[NNs]{Neuronale Netze}
% O
\acro{owl}[OWL]{Web Ontology Language}
% P
% Q
% R
\acro{rdf}[RDF]{Resource Development Framework}
% S
\acro{snik}[SNIK]{Semantisches Netz des Informationsmanagements im Krankenhaus}
\acro{sparql}[SPARQL]{SPARQL Protocol and RDF Query Language}
% T
% U
\acro{uri}[URI]{Uniform Resource Identifier}
\acroplural{uri}[URIs]{Uniform Resource Identifiers}
\acro{url}[URL]{Uniform Resource Locator}
\acroplural{url}[URLs]{Uniform Resource Locators}
% V
% W
\acro{w3c}[W3C]{World Wide Web Consortium}
\acro{www}[WWW]{World Wide Web}
% X
% Y
% Z
\end{acronym}
%\acused{URL}% Has its own paragraph in the preliminaries.
\acused{snik}% Explained in Related Work, but appears in Introduction & Preliminaries
\acused{dnn}% Written manually in Section 2.3.2
