%*******************************************************
% Abstract
%*******************************************************
\pdfbookmark[1]{Abstrakt}{Abstrakt}
\chapter*{Abstrakt}
\addcontentsline{toc}{chapter}{Abstrakt}
Mit der beständig fortschreitenden Digitalisierung im Gesundheitswesen wird es immer wichtiger, auch das Wissen über das Informationsmanagement dort digital und strukturiert erreichbar zu machen. In dieser Arbeit beschäftige ich mich mit der vom Institut für Medizinische Informatik, Statistik und Epidemiologie entwickelten Ontologie SNIK.
Diese fasst Wissen aus dem Bereich des Informationsmanagements im Krankenhaus und soll künftig auch bei dem Studium der Medizininformatik helfen. Momentan ist sie aber nicht mit geschriebener Sprache abrufbar, sondern nur über andere Werkzeuge, welche es schwierig machen, die Informationen zu durchsuchen. Eine potentielle Lösung hierfür ist die Fragenbeantwortung. Es soll möglich sein, dass ein Nutzer eine englische Frage in Satzform stellt und darauf eine Antwort bekommt. Hierfür gibt es verschiedene Systeme, viele sind allerdings auf andere Ontologien spezialisiert. Die Schwierigkeit bei solchen Systemen ist einerseits, die menschlichen Fragen in eine für Computer lesbare Form, d.h. eine Abfrage an die Ontologie, umzuformen und andererseits, aus verschiedenen Antwortmöglichkeiten die auszuwählen, welche die fragende Person am wahrscheinlichsten meinte.
Das Ziel dieser Arbeit ist, nach Systemen zur Fragenbeantwortung zu recherchieren und letztendlich eines auf SNIK anzuwenden. Die Antworten sollen außerdem anhand eines vorher definierten Fragenkataloges auf ihre Genauigkeit hin überprüft und gewertet werden.
\vfill
